% !TEX root = ../main.tex

% ----------------------------------
% labels: \label{conc:[type]:[name]}
% ----------------------------------
% PAST TENSE


% \textcolor{red}{
% \begin{itemize}
%     \item Weakness to our code: Inconsistency related to $N\ped{eff}$
%     \item Did not include neutrinos, helium, reionisation or polarisation---any of that which create interesting results
%     \item Further improvements: Include neutrinos, helium, reionisation and polarisation
% \end{itemize}
% }


We used fiducial parameters in the \textLambda CDM concordance model of cosmology to calculate the CMB anisotropy spectrum. The preparations from~Sects.~\ref{sec:mil1} \&~\ref{sec:mil2} lead to vanilla results in~Sects.~\ref{sec:mil3} \&~\ref{sec:mil4} that were satisfactory, more so on large scales. We did not expect the final results to be perfectly coherent with observational data, but rather use them to understand the physical mechanisms we considered in out model.

We acknowledge the inconsistency from the ambiguous $N\ped{eff}$-value in this paper. In refusing to perturb neutrino-related quantities, one also fiddles with the neutrino-keeping background in an unfortunate way, so they should be ignored completely.

Further improvements include rewriting parts of the code to account for secondary physical mechanisms. Particularly we would add helium and followingly reionisation in~\cref{sec:mil2} and account for neutrino perturbations and photon polarisation in~\cref{sec:mil3}. To get more hands-on experience with the physical implications of the CMB anisotropy, one should experiment with parameters systematically.