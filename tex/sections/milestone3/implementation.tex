% !TEX root = ../../main.tex

% --------------------------------------
% labels: \label{mil3:imp:[type]:[name]}
% --------------------------------------
% PAST TENSE


We extended our \verb|C++| program to account for perturbations in order to study the evolution of structure in the universe. Objects of the classes from~\cref{mil1:sec:imp} and~\cref{mil2:sec:imp} was sufficient for the newest class to do all the desired work. 



The following recipe \ref{item_one} -- \ref{item_six} was used for each of the 100 logarithmically spaced values $k \in [5\cross10^{-5}, \,3\cross 10^{-1}]\unit{Mpc}^{-1}$. We used an array of $5\cross 10^4+1$ values of $x\in [x\ped{init}, 0]$ with $x\ped{init}=-20$.

\begin{enumerate}[wide,labelwidth=!,labelindent=0pt,label=(\roman*)]
    \item\label{item_one} 
    For the initial time $x\ped{init}$, set the initial conditions according to~\cref{app:pert:sec:initial_conditions} (\cref{app:pert:eq:initial_conditions} and~\eqref{app:pert:eq:initial_conditions_Theta01}). 

    \item\label{item_two} 
    
    Compute the time for which tight coupling ends, $x\ped{tc,end}(k)$. That is, find the $x$ for which~\cref{mil3:theo:eq:tc_condition_one},~\eqref{mil3:theo:eq:tc_condition_two} and/or~\eqref{mil3:theo:eq:tc_condition_three} is violated. 
    
    \item\label{item_three}
    
    Solve the set of differential equations from~\cref{app:pert:sec:tight_coupling} using RK4. Integrate from $x=x\ped{init}$ to $x=x\ped{tc,end}(k)$.

    \item\label{item_four}
    
    Use the result from the tight coupling regime at $x=x\ped{tc,end}(k)$ to set the initial conditions for the full system. The higher multipoles are set according to~\cref{app:pert:eq:initial_conditions_Theta2ell} with $x=x\ped{tc,end}(k)$.

    \item\label{item_five}
    
    Solve the set of differential equations from~\cref{app:pert:sec:full_system} using RK4. Integrate from $x=x\ped{tc,end}(k)$ to $x=0$. 

    \item\label{item_six}
    
    Sew solutions together and save result.

\end{enumerate}


The expression for $\Sti(k,x)$ as provided in~\cref{app:app:source} was implemented succeeding a successful computation of the perturbations.



\subsubsection{Parallelisation}\label[sec]{mil3:imp:sec:openmp}
    The set of equations is completely decoupled in $k$, making our main problem highly parallelisable. Very few, very simple lines were added to code to execute \ref{item_one} -- \ref{item_six} in parallel using shared-memory parallelisation according to the \textit{OpenMP} standard.
