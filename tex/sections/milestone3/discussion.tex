% !TEX root = ../../main.tex

% ---------------------------------------
% labels: \label{mil3:disc:[type]:[name]}
% ---------------------------------------
% PRESENT TENSE





% The ``shape'' of the first two moments of the photon temperature are similar for all scales, but the onset of oscillations happens later for smaller $k$. This is clear from the lower panels of~\cref{mil3:res:fig:matter_perturbations}.

%However, turning to the lower panel, the other scalar potential $\Psi(x, k)$ must have oscillating properties as the sum of the two potentials goes as damped oscillators. For the large-scale mode, this oscillator is 





% From the upper panels of \cref{mil3:res:fig:matter_perturbations}, it is clear that only the small-scale modes experience oscillations in the baryonic matter perturbations. 

% We consider~\cref{mil3:res:fig:density_perturbations}. As expected, the overdensities for CDM and baryons grow over time due to gravitational collapse, while the photon density perturbation undergoes oscillations due to the tight coupling between photons and baryons before recombination. For all species, the amplitude of the density perturbation is suppressed on large-scales ($k \sim k_l$) due to acoustic oscillations in the early universe, while on small-scales ($k \sim k_s$), the amplitude is enhanced due to gravitational collapse. 


The upper panels of \cref{mil3:res:fig:matter_perturbations} show that only the small-scale modes experience oscillations in the baryonic matter perturbations. In \cref{mil3:res:fig:density_perturbations}, we observe that the overdensities of cold dark matter and baryons grow over time due to gravitational collapse, while the photon density perturbation undergoes oscillations due to the tight coupling between photons and baryons before recombination. On large scales ($k \sim k_l$), the amplitude of the density perturbation is suppressed due to acoustic oscillations in the early universe, while on small scales ($k \sim k_s$), the amplitude is enhanced due to gravitational collapse.

% The tight coupling between photons as baryons is clear from the upper panel of~\cref{mil3:res:fig:velocity_perturbations}. For the small-scale mode, the photon velocity starts deviating from the baryon velocity shortly after $x=x\ped{eq}$. For the intermediate-scale mode, this seems to happen around the same time $x\sim-7$, but a little later on. The same happens for the large-scale mode at $x\sim -5$, so a significant while later. Also visible from the same plots, are the approximate time of horizon crossing for the different wavenumbers. We see this as the time the photon velocity starts deviating from the velocity of the cold dark matter---at $x\sim -11$, $-8$ and $-5$ for $k_s$, $k_i$ and $k_l$, respectively. On large-scales, the bulk velocities are suppressed due to the adiabatic initial conditions, while on small-scales, they are enhanced due to gravitational collapse. The bulk velocity for photons undergoes oscillations due to the same acoustic physics as the photon overdensity.

The tight coupling between photons and baryons is evident from the upper panel of \cref{mil3:res:fig:velocity_perturbations}. For the small-scale mode, the photon velocity starts deviating from the baryon velocity shortly after $x=x\ped{eq}$. For the intermediate-scale mode, this happens around the same time $x\sim-7$, but a little later on. The same occurs for the large-scale mode at $x\sim -5$, a significant while later. We also observe the approximate time of horizon crossing for the different wavenumbers, which we see as the time when the photon velocity starts deviating from the velocity of the cold dark matter---at $x\sim -11$, $-8$, and $-5$ for $k_s$, $k_i$, and $k_l$, respectively. On large scales, the bulk velocities are suppressed due to adiabatic initial conditions, while on small scales, they are enhanced due to gravitational collapse. The bulk velocity for photons undergoes oscillations due to the same acoustic physics as the photon overdensity.



% Moving on to the quadrupole moment $\Theta_2(x,k)$ in~\cref{mil3:res:mil3:res:fig:photon_quadrupole}, we see that this quantity exhibits oscillatory behaviour on all scales due to the same acoustic physics as the photon overdensity and bulk velocity. On small-scales, the amplitude of the oscillations is suppressed as photons diffuse from overdense regions.


The quadrupole moment $\Theta_2(x,k)$ in \cref{mil3:res:fig:photon_quadrupole} exhibits oscillatory behaviour on all scales due to the same acoustic physics as the photon overdensity and bulk velocity. For small-scale modes, the amplitude of the oscillations is suppressed as photons diffuse from overdense regions.

% We study the upper panel of~\cref{mil3:res:fig:gravitational_potential}. The scalar potential of mode that enters the horizon during the radiation-dominated era, $\Phi(x,k\!=\!k_s)$, drops as soon as it enters the horizon and starts oscillating. This does not happen for the two larger scales. Turning to the lower panel, what is essentially plotted is $\propto -\Theta_2(x, k)/k^2\cdot \eu[-2x]$ (see~\cref{app:pert:eq:Psi}). After recombination, the anisotropic stress becomes non-zero and so does the sum $\Psi + \Phi$ for all scales.

In the upper panel of \cref{mil3:res:fig:gravitational_potential}, we study the scalar potential of the mode that enters the horizon during the radiation-dominated era, $\Phi(x,k\!=\!k_s)$. As soon as it enters the horizon, it drops and starts oscillating, which does not happen for the two larger scales. The lower panel shows a plot of $\propto -\Theta_2(x, k)/k^2\cdot \eu[-2x]$ (see \cref{app:pert:eq:Psi}). After recombination, the anisotropic stress becomes non-zero, and so does the sum $\left(\Psi + \Phi\right)(x,k)$ for all scales.


% In summary, the behaviour of cosmic perturbations depends strongly on scale, with different physical processes dominating on large- and small-scales. Our results are consistent with expectations based on standard cosmological models and provide a valuable insight into the growth of structure in the universe.

In summary, cosmic perturbations exhibit scale-dependent behaviour, with different physical processes dominating on large and small scales. Our results are consistent with expectations based on standard cosmological models and provide valuable insight into the growth of structure in the universe.


% Finally, we considered the spatial curvature $\Phi(x,k)$ and the sum of this quantity with the gravitational potential $\Psi(x,k)$. On large-scales, both $\Phi(x,k_l)$ and $\Psi(x,k_l)$ are approximately constant, while on small-scales, they exhibit oscillatory behavior due to acoustic physics and are enhanced due to gravitational collapse.