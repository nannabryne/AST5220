% !TEX root = ../../main.tex

% ---------------------------------------
% labels: \label{mil3:disc:[type]:[name]}
% ---------------------------------------
% PRESENT TENSE





% The ``shape'' of the first two moments of the photon temperature are similar for all scales, but the onset of oscillations happens later for smaller $k$. This is clear from the lower panels of~\cref{mil3:res:fig:matter_perturbations}.

%However, turning to the lower panel, the other scalar potential $\Psi(x, k)$ must have oscillating properties as the sum of the two potentials goes as damped oscillators. For the large-scale mode, this oscillator is 





% From the upper panels of~\cref{mil3:res:fig:matter_perturbations}, it is clear that only the small-scale modes experience oscillations in the baryonic matter perturbations. 

% We consider~\cref{mil3:res:fig:density_perturbations}. As expected, the overdensities for CDM and baryons grow over time due to gravitational collapse, while the photon density perturbation undergoes oscillations due to the tight coupling between photons and baryons before recombination. For all species, the amplitude of the density perturbation is suppressed on large-scales ($k \sim k_l$) due to acoustic oscillations in the early universe, while on small-scales ($k \sim k_s$), the amplitude is enhanced due to gravitational collapse. 

What was not discussed too much in~\cref{mil3:sec:theo}, was the physics of the acoustic oscillations and how they are frozen in at decoupling. We shall study these mechanisms in more detail in~\cref{mil4:sec:theo}. However, we choose to consider some of these concepts when discussing the results in this section, hoping that the statements will be justified later on.


To summarise the discussion below, cosmic perturbations exhibit scale-dependent behaviour, with different physical processes dominating on large and small scales. Our results are consistent with expectations based on standard cosmological models and provide valuable insight into the growth of structure in the universe.


\paragraph{Matter perturbations}
The upper panels of~\cref{mil3:res:fig:matter_perturbations} show that only the small-scale modes experience oscillations in the baryonic matter perturbations. Modes on larger scales than this enter the horizon too late to be affected by the causal physics that is the interplay between the radiation pressure and the gravitational force in the photon-baryon plasma---that which gives rise to the acoustic oscillations. The adiabatic initial conditions make sure the normal matter perturbations are the same for CDM and baryons in the beginning and cease to be the same in the end, for all modes. Although for large-scale modes, these species never decouple at all since horizon entry happens in proper matter domination. Shortly after the onset of recombination, sub-horizon baryonic matter falls into CDM potential wells, demonstrated by the baryon overdensities during the end of the Compton drag epoch (i.e. $\delta\ped{b}(x\!\approx\!x_*, k\!=\!k_s, k_i)$ in~\cref{mil3:res:fig:density_perturbations}). 

In~\cref{mil3:res:fig:density_perturbations}, we observe that the overdensities of (non-interacting) cold dark matter inside the horizon grows exponentially in the matter dominated era. During radiation domiation, the sub-horizon CDM overdensity, $\delta\ped{c}(x, k_s)$, grows even faster. On super-horizon scales, the CDM overdensities are are constant, as a result of our gauge choice.\footnote{$\delta\ped{c}(x, k\!\ll\! \Hp) \propto \mathcal{R}|_{k\ll \Hp} = \text{constant}$ in the Newtonian gauge.} In addition, we can see that the overdensities flatten out as we enter the DE dominated era ($x_\Lambda\approx -0.026$), which is expected from the SW effect.

In the fully recomined universe as well as on super-horizon scales, the baryonic matter perturbations reduce to those of dark matter. This is not completely accurate for $\delta\ped{b}(x\!\ga\!x_*,k_s)$, c.f.~\cref{mil3:res:fig:density_perturbations}, upper panel. Here, the state at which the mode $k_s$ is caught in at recombination, is ``just starting is sixth decompression'' since $k_s\approx 5.2 k_1$, c.f.~\cref{mil3:res:tab:wavenumbers} (or simply count compressions and decompressions in $\Theta_0(x\!<\!x_*, k_s)$). 

Even though $k_i$ is inside the horizon at $x=x_*$, it does not probe baryonic acoustic oscillations (BAOs) as the fluid system is only half-way to its first extremum. %\textcolor{blue}{check this!}
%That is to say, the system in question is  when photons ``suddenly'' decouple,

We spot indications of \textit{baryon loading} in the lower panel of~\cref{mil3:res:fig:density_perturbations}. This concept will be introduced in~\cref{mil4:theo:sec:eff_temp_pert}. 

% It is worth noting the 
%\textcolor{blue}{Comment about photon rattling and \textit{drag epoch}??}



% The photon density and velocity perturbations follow the matter perturbations until decoupling, and the photons stream freely after this.

% In~\cref{mil3:res:fig:density_perturbations}, we observe that the overdensities of cold dark matter and baryons grow over time due to gravitational collapse, while the photon density perturbation undergoes oscillations due to the tight coupling between photons and baryons before recombination. On large scales ($k \sim k_l$), the amplitudes of the density perturbations are suppressed due to acoustic oscillations in the early universe, while on small scales ($k \sim k_s$), the amplitude is enhanced due to gravitational collapse. 



% The tight coupling between photons as baryons is clear from the upper panel of~\cref{mil3:res:fig:velocity_perturbations}. For the small-scale mode, the photon velocity starts deviating from the baryon velocity shortly after $x=x\ped{eq}$. For the intermediate-scale mode, this seems to happen around the same time $x\sim-7$, but a little later on. The same happens for the large-scale mode at $x\sim -5$, so a significant while later. Also visible from the same plots, are the approximate time of horizon crossing for the different wavenumbers. We see this as the time the photon velocity starts deviating from the velocity of the cold dark matter---at $x\sim -11$, $-8$ and $-5$ for $k_s$, $k_i$ and $k_l$, respectively. On large-scales, the bulk velocities are suppressed due to the adiabatic initial conditions, while on small-scales, they are enhanced due to gravitational collapse. The bulk velocity for photons undergoes oscillations due to the same acoustic physics as the photon overdensity.


\paragraph{Photon temperature perturbations}
The tight coupling between photons and baryons is evident from their respective velocity graphs in th upper panel of~\cref{mil3:res:fig:velocity_perturbations}. For the sub-horizon modes ($k_s$ and $k_l$), the photon velocity starts deviating from the baryon velocity during recombination, at which the high-frequency photon starts rattling around before it streams freely in the universe.

On large scales, the bulk velocities are suppressed due to adiabatic initial conditions, while on small scales, they are enhanced due to gravitational collapse. The bulk velocity for photons undergoes oscillations due to the same acoustic physics as the photon overdensity.

We also observe the approximate time of horizon crossing for the different wavenumbers, which we see as the time when the photon velocity starts deviating from the velocity of the cold dark matter.

Finally, we observe that for all modes, as they enter the horizon, they first compress ($\dv*{\delta_\gamma}{x}> 0$, $u_\gamma> 0$) once before decompressing, then compressing again, and so on. 


% \dots
% % For the intermediate-scale mode, this release happens around the same time $x\sim-7$, but a little later on. The same occurs for the large-scale mode at $x\sim -5$, a significant while later. 
% % We also observe the approximate time of horizon crossing for the different wavenumbers, which we see as the time when the photon velocity starts deviating from the velocity of the cold dark matter---at $x\sim -11$, $-8.5$, and $-5.2$ for $k_s$, $k_i$, and $k_l$, respectively. 
% On large scales, the bulk velocities are suppressed due to adiabatic initial conditions, while on small scales, they are enhanced due to gravitational collapse. The bulk velocity for photons undergoes oscillations due to the same acoustic physics as the photon overdensity.
% \dots


% Moving on to the quadrupole moment $\Theta_2(x,k)$ in~\cref{mil3:res:mil3:res:fig:photon_quadrupole}, we see that this quantity exhibits oscillatory behaviour on all scales due to the same acoustic physics as the photon overdensity and bulk velocity. On small-scales, the amplitude of the oscillations is suppressed as photons diffuse from overdense regions.


The quadrupole moment $\Theta_2(x,k)$ in~\cref{mil3:res:fig:photon_quadrupole} exhibits oscillatory behaviour on all scales due to the same acoustic physics as the photon overdensity and bulk velocity. For small-scale modes, the amplitude of the oscillations is suppressed as photons diffuse from overdense regions. This is an example of \textit{diffusion damping} that will be discussed later. 

% We study the upper panel of~\cref{mil3:res:fig:gravitational_potential}. The scalar potential of mode that enters the horizon during the radiation-dominated era, $\Phi(x,k\!=\!k_s)$, droxps as soon as it enters the horizon and starts oscillating. This does not happen for the two larger scales. Turning to the lower panel, what is essentially plotted is $\propto -\Theta_2(x, k)/k^2\cdot \eu[-2x]$ (see~\cref{app:pert:eq:Psi}). After recombination, the anisotropic stress becomes non-zero and so does the sum $\Psi + \Phi$ for all scales.

\paragraph{Metric perturbations}
We make a note of the fact that the curvature perturbation is essentially a perturbation to the expansion factor, $\Phi \cor \delta a /a$. In the upper panel of~\cref{mil3:res:fig:gravitational_potential}, we study the scalar potential of the mode that enters the horizon during the radiation-dominated era, $\Phi(x,k\!=\!k_s)$. Horizon entry provokes the growth of matter perturbations and the potential drops. This is because radiation has pressure and therefore does not manage to cluster enough to fight the universe's expansion and keep the potentials up. As matter starts to dominate radiation, the propagating acoustic wave (in real space) generated by the oscillating baryonic matter (in Fourier space) is imprinted in the gravitational potential. We see this as the oscillations starting at $x \sim x\ped{eq}$ in~\cref{mil3:res:fig:gravitational_potential}.

Variations gravitational potentials exhibit large-scale physical effects and therefore also affect super-horizon scales. Around $x \sim x\ped{eq}$ expansion is affected by radiation and matter alike, resulting in decaying gravitational potentials up until any residual radiation has vanished, where they flatten out. The fact that the baryon fraction $\Omega\ped{b0}/\OmgM$ is substantial manifests in the further decay of the potentials between radiation-matter equality and the end of the Compton drag epoch, due to their inability to cluster below the sound horizon. Intermediate scales experience more or less the same potential evolutions as large scales. We can see how the DE at late times accelerates the expansion and causes large-scale potential wells/hills to decay (c.f. $\Phi(x\ga x_\Lambda, k\!=\!k_i, k_l)$). %This is called the ``late-time'' ISW effect.
% and starts oscillating, which does not happen for the two larger scales. 

% The oscillations in Fourier space manifest the propagating wave \dots
% \textcolor{orange}{
% For the $k_l$- and $k_i$-modes, this potential is more or less constant in the radiation- and matter-dominated eras, having a smaller value in the latter. This is consistent with predictions from e.g.~\citet{Baumann}.
% }

As a side note, the lower panel shows a plot of $\propto -\Theta_2(x, k)/k^2\cdot \eu[-2x]$ (see~\cref{app:pert:eq:Psi}). After recombination, the anisotropic stress becomes non-zero, and so does the sum $\left(\Psi + \Phi\right)(x,k)$ for all scales, as we anticipated. Tiny fluctuations are visible before this for the small-scale mode.
\footnote{Most evident when scaling the graphs with the wavenumber, e.g.~$k^{\sfrac{3}{2}} \left(\Psi + \Phi \right)$.}


% In summary, the behaviour of cosmic perturbations depends strongly on scale, with different physical processes dominating on large- and small-scales. Our results are consistent with expectations based on standard cosmological models and provide a valuable insight into the growth of structure in the universe.


% Finally, we considered the spatial curvature $\Phi(x,k)$ and the sum of this quantity with the gravitational potential $\Psi(x,k)$. On large-scales, both $\Phi(x,k_l)$ and $\Psi(x,k_l)$ are approximately constant, while on small-scales, they exhibit oscillatory behavior due to acoustic physics and are enhanced due to gravitational collapse.