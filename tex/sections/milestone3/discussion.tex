% !TEX root = ../../main.tex

% ---------------------------------------
% labels: \label{mil3:disc:[type]:[name]}
% ---------------------------------------
% PRESENT TENSE





The tight coupling between photons as baryons is clear from the upper panel of~\cref{mil3:res:fig:velocity_perturbations}. For the small-scale mode, the photon velocity starts deviating from the baryon velocity shortly after $x=x\ped{eq}$. For the intermediate-scale mode, this seems to happen around the same time $x\sim-7$, but a little later on. The same happens for the large-scale mode at $x\sim -5$, so a significant while later.

From the upper panels of \cref{mil3:res:fig:matter_perturbations}, it is clear that only the small-scale experience oscillations in the baryonic matter perturbations. However, there is a slight tendency visible for the intermediate-scale in the velocity perturbation.

The ``shape'' of the first two moments of the photon temperature are similar for all scales, but the onset of oscillations happens later for smaller $k$. This is clear from the lower panels of~\cref{mil3:res:fig:matter_perturbations}.

We study the upper panel of~\cref{mil3:res:fig:gravitational_potential}. The scalar potential of mode that enters the horizon during the radiation-dominated era $\Phi(k\!=\!k_s,x)$ drops as soon as it enters the horizon and starts oscillating. This does not happen for the two larger scales. Turning to the lower panel, what we essentially are plotting is $\propto -\Theta_2(k,x)/k^2\cdot \eu[-2x]$ (see~\cref{app:pert:eq:Psi}).

%However, turning to the lower panel, the other scalar potential $\Psi(k,x)$ must have oscillating properties as the sum of the two potentials goes as damped oscillators. For the large-scale mode, this oscillator is 

