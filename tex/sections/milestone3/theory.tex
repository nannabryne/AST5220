% !TEX root = ../../main.tex

% ---------------------------------------
% labels: \label{mil3:theo:[type]:[name]}
% ---------------------------------------
% PRESENT/FUTURE TENSE



The geometry of the spacetime is encoded in the Einstein tensor $G\munu$ that relates to the energy-momentum tensor $T\munu$ through the Einstein equation $G\munu = 8\pi G T\munu$. The latter quantity describes the distrubution of energy and matter. The perturbed Einstein equation can be written
\begin{equation}
    \overline{G}\munu +\delta G\munu = 8\pi G \left( \overline{T}\munu+\delta T_{\mu \nu} \right),
\end{equation}
where we use (and will continue to use) an overline to indicate the unperturbed quanity. Being primarily interested in the perturbations, it all boils down to solving
\begin{equation}\label{mil3:theo:eq:delta_G}
    \delta G\munu = 8\pi G\, \delta T\munu
\end{equation}
in a clever way.\footnote{Solving it directly is highly non-trivial.} The metric is related to the Einstein tensor through $G\munu= R\munu - \frac{1}{2}g\munu R^\rho_{\, \rho}$, where $R\munu$ is the Ricci tensor. 
%We will solve differential equations for:
% \begin{itemize}
%     \item spacetime corrections (potentials) $\Psi$ and $\Phi$
%     \item density perturbations to normal matter $\delta\ped{c}$ and $\delta\ped{b}$
% \end{itemize}
Below is presented a short description of the quantities we aim to compute. 
\begin{itemize}
    \item $\Psi$ and $\Phi$ are spacetime corrections (potentials), both closely related to the Newtonian gravitational potential.
    \item $\delta\ped{c}$ and $\delta\ped{b}$ are density perturbations to non-relativistic matter, representing cold dark matter and baryonic matter, respectively.
    \item $u\ped{c}$ and $u\ped{b}$ are velocity perturbations to non-relativistic matter, representing cold dark matter and baryonic matter, respectively.
    \item $\Theta_{\ell= 0, 1, \dots, \ell\ped{max}}$ are the photon multipoles that show up when expanding the photon temperature fluctuations $\Theta$ into spherical harmonics.
\end{itemize}

We will not go into detail in how to get from \cref{mil3:theo:eq:delta_G} to the set of equations we will eventually use, but rather keep in mind that the quantities $\delta\ped{c,b}$, $u\ped{c,b}$, $\Theta_\ell$, $\Phi$ and $\Psi$ are deduced from the Einstein equation and first order linear perturbation theory. 


\subsubsection{Metric perturbations}\label[sec]{mil3:theo:sec:metric_pert}
    The FRW metric from \cref{mil1:theo:eq:FRW} describes a smooth universe. A first order linear perturbation to the metric can be applied through
    \begin{equation}\label{mil3:theo:eq:xxxxxx}
        g\munu = \overline{g}\munu + h\munu,
    \end{equation}
    where $\overline{g}\munu $ is the zeroth order term---the FRW metric.\footnote{We use the term ``metric'' on both $g\munu$ and $ds^2$. Recall $ds^2 = g\munu\diff x^\mu \diff x^\nu$.} We will work the the so-called conformal-Newtonian gauge, where the perturbation $h\munu$ is
    \begin{equation}
    \begin{split}
        h_{00} &= 2\Psi \\
        h_{0i} &= 0 \\
        h_{ij} &= 2\Phi \deltasym{ij}.
    \end{split}
    \end{equation}
    The two potentials $\Psi$ and $\Phi$ are closely related to the Newtonian gravitational potential. Our perturbed metric is now given by
    \begin{equation}
    \begin{split}
        g_{00} &= - \left(1+2\Psi\right) \\
        g_{0i} &= 0 \\
        g_{ij} &= \eu^{2x}\deltasym{ij}\left(1+2\Phi \right).
    \end{split}
    \end{equation}


    % It can be shown \citep[see e.g.][Ch.~4]{Callin2006} that writing
    % \begin{equation}
    %     ds^2 = - (1-)
    % \end{equation}




\subsubsection{Temperature fluctuations}\label[sec]{mil3:theo:sec:temperature}
    The photon temperature $T\gped{\textgamma}$ \dots
    \begin{equation}
        \Tgamma (x) = \overline{T}\gped{\textgamma}(x) + \delta \Tgamma(x)= \TCMB \eu^{-x} \left[1 + \Theta(x)\right]
    \end{equation}



\please

\noindent\colorbox{orange}{FULL SYSTEM:}
\par \textcolor{blue}{\underline{Metric perturbations}}
\begin{subequations}\label{mil3:theo:eq:metric_perturbations}
\begin{align}
    \dv{\Phi}{x} &= \Psi - \frac{c^2 k^2}{3\Hp^2} \Phi + \frac{H_0^2}{2\Hp^2\eu[2x]}\left\{ \left(\Omega\ped{c 0}\delta\ped{c} + \Omega\ped{b 0}\delta\ped{b} \right)\eu[x]+4\Omega\gped{\textgamma 0}\Theta_0\right\} \\
    \Psi &= - \Phi - \frac{12H_0^2}{c^2k^2\eu^{2x}} \Omega\gped{\textgamma 0}\Theta_2 
\end{align}
\end{subequations}

\par \textcolor{blue}{\underline{Matter perturbations}}
\begin{subequations}\label{mil3:theo:eq:matter_perturbations}
\begin{align}
    \dv{\delta\ped{c, b}}{x} &=  \ckH u\ped{c, b} - 3\dv{\Phi}{x}  \\
    \dv{u\ped{c, b}}{x} &= - u\ped{c, b} - \ckH\Psi + \mathbf{1}\ped{b} \dv{\tau}{x} \frac{3\Theta_1 +u\ped{b}}{R(x)} 
\end{align}
\end{subequations}

\par \textcolor{blue}{\underline{Photon temperature perturbations}}
% \begin{equation}\label{mil3:theo:eq:photontemp_perturbations}
% \dv{\Theta_\ell}{x} = \begin{cases}
%     - \ckH\Theta_{\ell+1}- \dv{\Phi}{x} & \ell =0\\
%     \frac{ck}{3\Hp} \left( \Theta_{\ell-1} - 2\Theta_{\ell+1} + \Psi\right) + \dv{\tau}{x}\left[\Theta_{\ell} + \frac{1}{3}u\ped{b} \right] & \ell =1  \\
%     \frac{ck}{(2\ell +1)\Hp} \left(\ell \Theta_{\ell-1}- (\ell + 1)\Theta_{\ell+1} \right) +\frac{9}{10} \dv{\tau}{x}\Theta_{\ell} &  \ell =2 \\
%     \frac{ck}{(2\ell +1)\Hp} \left(\ell \Theta_{\ell-1}- (\ell + 1)\Theta_{\ell+1} \right) + \dv{\tau}{x}\Theta_\ell  & 3 \leq \ell < \ell\ped{max} \\
%     \ckH \Theta_{\ell-1}- \frac{c(\ell+1)}{\Hp \eta}\Theta_{\ell} + \dv{\tau}{x}\Theta_\ell &  \ell = \ell\ped{max}
% \end{cases}
% \end{equation}


% \begin{equation}
% \begin{split}
%     \dv{\Theta_\ell}{x} = &\frac{ck\left(\ell \Theta_{\ell-1} - (\ell+1)\Theta_{\ell+1}\right)}{(2\ell +1) \Hp}  + \dv{\tau}{x} \Theta_\ell \left(1-\frac{\delta_{\ell,2}}{10}\right) + \mathcal{K}
% \end{split}
% % \left[\Theta_\ell + \frac{\delta_{\ell,1} u\ped{b}}{3} - \frac{\delta_{\ell,2}\Theta_\ell}{10} \right] + \mathcal{K}
% \end{equation}
% \begin{equation}
%     \mathcal{K} = \begin{cases}
%         -\dv{\Phi}{x} &\ell = 0 \\
%         \frac{ck}{3\Hp}\Psi + \dv{\tau}{x}\frac{u\ped{b}}{3} & \ell =1 \\
%         0 & \ell \leq 2
%     \end{cases}
% \end{equation}

% \begin{equation}
% \begin{split}
%     \dv{\Theta_\ell}{x} = &\frac{ck\left[\ell \Theta_{\ell-1} - (\ell+1)\Theta_{\ell+1}\right]}{(2\ell +1) \Hp}  + \dv{\tau}{x} \Theta_\ell \\%\left(1-\frac{\delta_{\ell,2}}{10}\right) \\
%     &- \delta_{\ell,0}\dv{\Phi}{x} + \delta_{\ell,1}\left[\frac{ck}{3\Hp}\Psi + \dv{\tau}{x}\frac{u\ped{b}}{3}\right] - \delta_{\ell,2} \dv{\tau}{x}\frac{\Theta_\ell}{10}
% \end{split}
% \end{equation}
\begin{subequations}
\par \textcolor{blue}{$\ell = 0$ --- FIXME!!} 
\begin{equation}
    \dv{\Theta_{\ell= 0}}{x} = \frac{ck}{\Hp}\Theta_{\ell+1}  -\dv{\Phi}{x}
\end{equation}
\par \textcolor{blue}{$\ell < \ell\ped{max}$ --- FIXME!!} 
\begin{equation}
\begin{split}
    \dv{\Theta_{0<\ell< \ell\ped{max}}}{x} = &\frac{ck\left[\ell \Theta_{\ell-1} - (\ell+1)\Theta_{\ell+1}+\delta_{\ell,1} \Psi\right]}{(2\ell +1) \Hp}  \\
    &+ \dv{\tau}{x}\left[\left(1 -\frac{\delta_{\ell,2}}{10}\right)\Theta_\ell+ \frac{\delta_{\ell,1} u\ped{b}}{3}\right] 
\end{split}
\end{equation}
\par \textcolor{blue}{$\ell = \ell\ped{max}$}
\begin{equation}
    \dv{\Theta_{\ell = \ell\ped{max}}}{x} = \ckH \Theta_{\ell-1} - \frac{c(\ell+1)}{\Hp \eta}\Theta_{\ell} + \dv{\tau}{x} \Theta_{\ell}
\end{equation}
\end{subequations}
% \begin{equation}\label{mil3:theo:eq:photontemp_perturbations}
% \dv{\Theta_\ell}{x} = \frac{ck}{(2\ell +1)\Hp} \left(\ell \Theta_{\ell-1}- (\ell + 1)\Theta_{\ell+1} \right) + \dv{\tau}{x}\Theta_\ell


% \end{equation}



\noindent\colorbox{orange}{TIGHT COUPLING REGIME:}

\begin{subequations}\label{mil3:theo:eq:tight_regime}

\begin{equation}
\begin{split}
    q = &\left[\dv{\tau}{x}~(1+R) - R\left(1-\frac{1}{\Hp}\dv{\Hp}{x}\right)\right]^{-1} \cross \\
    \,&\Bigg\{ \left[ \dv[2]{\tau}{x}~(1+R) - \dv{\tau}{x}~(1-R) \right] (3\Theta_1 + u\ped{b}) \\
    \,&  - \ckH R\left[  \Psi + \left(1-\frac{1}{\Hp}\dv{\Hp}{x}\right)\left(-\Theta_0+ 2\Theta_2\right)- \dv{\Theta_0}{x}\right]
    \Bigg\}
\end{split}
\end{equation}
\begin{equation}
\begin{split}
    \dv{u\ped{b}}{x} =& \frac{q-Ru\ped{b} + \ckH(-\Theta_0+2\Theta_2)}{1+R} - \ckH\Psi 
\end{split}
\end{equation}
\begin{equation}
\begin{split}
    \dv{\Theta_1}{x} =& \frac{1}{3}\left(q-\dv{u\ped{b}}{x}\right)
\end{split}
\end{equation}
\end{subequations}


% \begin{subequations}\label{mil3:theo:eq:initial_conditions}
% \begin{align}
%     \Psi     &= -\frac{2}{3}     \label{mil3:theo:eq:init_Psi}\\
%     \Phi     &= -\Psi         \label{mil3:theo:eq:init_Phi}\\
%     \delta\ped{c, b}&= -\frac{3}{2} \Psi     \label{mil3:theo:eq:init_delta}\\
%     u\ped{c, b}     &= -\frac{ck}{2\Hp} \Psi \label{mil3:theo:eq:init_u}\\
%     \Theta_0 &= -\frac{1}{2}\Psi \label{mil3:theo:eq:init_Theta0}\\
%     \Theta_1 &= +\frac{ck}{6\Hp}\Psi\label{mil3:theo:eq:init_Theta1}\\
%     \Theta_2 &= -\frac{8ck}{15\Hp} \left(\dv{\tau}{x}\right)^{-1}\Theta_1\label{mil3:theo:eq:init_Theta2}\\
%     \Theta_\ell &= -\frac{\ell}{2\ell+1} \frac{ck}{15\Hp} \left(\dv{\tau}{x}\right)^{-1}\Theta_{\ell-1}\label{mil3:theo:eq:init_Thetal}
% \end{align}
% \end{subequations}


\noindent\colorbox{orange}{INITIAL CONDITIONS:}
\begin{subequations}\label{mil3:theo:eq:initial_conditions}
\begin{align}
    \Psi     &= -\frac{2}{3}     \label{mil3:theo:eq:init_Psi}\\
    \Phi     &= -\Psi         \label{mil3:theo:eq:init_Phi}\\
    \delta\ped{c, b}&= -\frac{3}{2} \Psi     \label{mil3:theo:eq:init_delta}\\
    u\ped{c, b}     &= -\frac{ck}{2\Hp} \Psi \label{mil3:theo:eq:init_u}\\
    \Theta_\ell &=\begin{cases}
        -\frac{1}{2}\Psi     & \ell = 0  \\
        +\frac{ck}{6\Hp}\Psi  & \ell = 1 \\ 
        -\frac{8ck}{15\Hp\dv*{\tau}{x}}\Theta_1    & \ell = 2 \\ 
        -\frac{\ell}{2\ell+1} \frac{ck}{\Hp\dv*{\tau}{x}} \Theta_{\ell-1} &\ell \geq 3
    \end{cases} \label{mil3:theo:eq:init_Theta}
\end{align}
\end{subequations}

\sendhelp
