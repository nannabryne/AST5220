% !TEX root = ../../main.tex

% --------------------------------------
% labels: \label{mil3:res:[type]:[name]}
% --------------------------------------
% PAST TENSE



% We present results for three wave numbers $k$, each of which representing its own regime: 
% \begin{itemize}
%     \item $k=0.001\unit{Mpc}$ represents the large scales
% \end{itemize}



The figures below demonstrate our results for three wave numbers $k$. We chose $k=0.001\unit{Mpc}^{-1}$ to represent the large-scale modes and $k=0.1\unit{Mpc}^{-1}$ for the small-scales. The intermediate-scale modes are represented by $k=0.01\unit{Mpc}^{-1}$.

We present the absolute values of the matter perturbations in the upper panels of~\cref{mil3:res:fig:matter_perturbations}. The lower panels of said figure show the density and velocity perturbations for the photons.
% \onecolumn
\begin{figure*}[!ht]
    \begin{subfigure}{0.49\linewidth}
        \centering
        \includegraphics[width=\linewidth]{milestone3/density_perturbations.png} 
        \caption{\textcolor{orange}{CAPTION}}
    \label[fig]{mil3:res:fig:density_perturbations}
    \end{subfigure}
    \begin{subfigure}{0.49\linewidth}
        \centering
        \includegraphics[width=\linewidth]{milestone3/velocity_perturbations.png} 
        \caption{\textcolor{orange}{CAPTION}} 
    \label[fig]{mil3:res:fig:velocity_perturbations}
    \end{subfigure}
    \caption{\textcolor{orange}{CAPTION (matter perturbations)}}
\label[fig]{mil3:res:fig:matter_perturbations}
\end{figure*}
% \twocolumn


The photon quadrupole is plotted in~\cref{mil3:res:fig:photon_quadrupole}.
\begin{figure}[!ht]
    \centering
    \includegraphics[width=\linewidth]{milestone3/photon_quadrupole.png} 
    \caption{The graphs show the quadrupole moment $\Theta_2(k, x)$ as function of logarithmic expansion factor $x$ for three different wavenumbers $k$.} 
    \label[fig]{mil3:res:fig:photon_quadrupole}
\end{figure}

\cref{mil3:res:fig:gravitational_potential} shows the gravitational potential as well as the sum of this and \colorbox{blue}{\textcolor{orange}{XXX}} as functions of logarithmic expansion factor for our chosen wave numbers.
\begin{figure}[!ht]
    \centering
    \includegraphics[width=\linewidth]{milestone3/gravitational_potential.png} 
    \caption{The graphs show the metric perturbations as functions of logarithmic expansion factor $x$ for three different wavenumbers $k$. Upper panel: spatial curvature $\Phi(k, x)$. Lower panel: sum of the spatial curvature $\Phi(k, x)$ and the gravitational potential $\Psi(k,x)$.} 
    \label[fig]{mil3:res:fig:gravitational_potential}
\end{figure}

Note that we completely dismissed neutrinos and photon polarisation. That is, we sat the effective neutrino number to be zero, slightly changing the background from \cref{sec:mil1}.

The end of tight coupling turned out to be $x\ped{tc,end}(k)=-8.3$ for all three wavenumbers.
