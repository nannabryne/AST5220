% !TEX root = ../main.tex

% -----------------------------------
% labels: \label{intro:[type]:[name]}
% -----------------------------------
% FUTURE TENSE

% \textcolor{red}{
% \textbf{Remember to include the following in this section:}
% \begin{itemize}
%     \item GOAL: predict the CMB (and matter) fluctuations through the power spectrum, starting from first principles + learn about the various physical processes that is happening in order to explain the results
%     \item divided into four steps: two concerning background cosmology and two concerning perturbations and ... (?)
%     \item remember to state abbreviations, e.g. cosmic microwave background (CMB) 
%     \item notation/conventions follow ~\citet{DodelsonBook}
%     \item flat \textLambda CDM model 
% \end{itemize}}




% WORKIG INTRO

The overall purpose of this project is to produce a program that calculates the Cosmic Microwave Background (CMB) power spectrum and predict the CMB and matter fluctuations through it. We want to achieve this starting from first principles. A large part of this paper follows~\citet{Callin2006}.

We consider the concordance model of cosmology model that a Euclidean universe currently dominated by non-baryonic cold dark matter (CDM) and a cosmological constant (\textLambda), namely the (flat) \textLambda CDM model. The cosmological constant \textLambda\, is used as a moniker for dark energy (DE).~\citep{DodelsonBook}

There are in principle six free parameters to our \textLambda CDM universe, excluding the averages CMB temperature $\TCMB$. The homogenous background is determined by the density parameter of the baryonic matter $\Omega\ped{b0}$, the CDM $\Omega\ped{c0}$ and curvature $\Omega\ped{K0}$, in addition to the Hubble constant $H_0 =100h \unit{km} \unit{s^{-1}} \unit{Mpc^{-1}}$. To obtain power spectra, we need to include the primordial power spectrum defined by its amplitude $\mathcal{A}\ped{S}$ and spectral index $n\ped{S}$.
%Were we to include neutrinos, helium, reionisation and/or polarisation (spoiler: we will not)


The structure of this paper reflects the project flow which was divided into four parts: the setup of the (unperturbed) background geometry in~\cref{sec:mil1}, studying the recombination history in~\cref{sec:mil2}, perturbing said background in~\cref{sec:mil3} and finally calculating the CMB and matter power spectra in~\cref{sec:mil4}. Each section is equipped with its own short introduction, theoretical background, coding details, results and discussion. 

Complementary material to this paper can be found in our~\href{https://github.com/nannabryne/AST5220}{Github repository} at \url{https://github.com/nannabryne/AST5220}.