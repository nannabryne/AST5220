% !TEX root = ../../main.tex

% ---------------------------------------
% labels: \label{mil2:theo:[type]:[name]}
% ---------------------------------------
% PRESENT/FUTURE TENSE

The process that is keeping the photons ($\gamma$) coupled to electrons ($e^{-}$) in the early universe is called Thomson scattering and is the low-energy case of Compton scattering,
\begin{equation}\label{mil2:theo:eq:Compton_Thomson}
    \gamma + e^{-} \getsto \gamma + e^{-},
\end{equation}
and considered to be the major source of opacity in the early universe. Going forward in time, the universe expanded and therefore cooled down, eventually to the temperature for which the combination of electrons and protons ($p$)---the formation of neutral hydrogen atoms (\element{H})---was energetically favourable. As a result, the amount of free electrons (and protons), compared to the total number of baryons in the universe, rapidly decreased. As direct recombinations to the ground state are highly unlikely, a hydrogen atom generally arises from an electron in a high energy state that immediately decays to its ground state, emitting a photon in the process;
\begin{equation}\label{mil2:theo:eq:relevant_interaction}
    e^{-} + p \getsto \element{H}+\gamma.
\end{equation}
There are two main pathways from the first excited state ($n=2$) to the ground state ($n=1$):
\begin{itemize}
    \item $2p\to 1s$: decay through the emission of a Lyman-\textalpha~photon that is (almost exclusively) to be reabsorbed by another ground state hydrogen \footnote{\textcolor{blue}{Write about redshifted line etc.!}}
    \item $2s\to 1s$: through the very slow process of 2-photon decay
\end{itemize}
These basic principles are amongst those that Jim Peebles and collaborators~\citep[see][]{Peebles1968} adopted in their model of non-equilibrium recombination history of hydrogen, described by the differential equation for the free electron abundance we call the \textit{Peebles equation} (\cref{mil2:theo:eq:Peebles}). There are numerical difficulties when solving this equation for very early times---this is where we address the work of Meghnad Saha in 1920. The \textit{Saha equation} (\cref{mil2:theo:eq:org_Saha}) is applicable in systems of chemical equilibrium and relates the number densities of reactants to those of the products in a reaction (assuming the equilibrium density of each element is known). At early times, this is a viable assumption, and we can examine the abundance of free electrons in the universe at all times.~\citep{Peebles1968,ChungPei1995,Callin2006}

The aforementioned equations originate from the Boltzmann equation, which can take the general form
\begin{equation}\label{mil2:theo:eq:Boltzmann}
    \frac{1}{n_1 \eu^{3x}} \dv{\left(n_1\eu^{3x}\right)}{x} = - \frac{\Gamma_1}{\Hp(x)\eu^{-x}} \left[1- \frac{n_3 n_4}{n_1 n_2} \frac{n_1^{(0)} n_2^{(0)}}{n_3^{(0)} n_4^{(0)}}  \right]
\end{equation}
for a reaction $(1) + (2) \getsto (3) + (4)$, where $n_j$ ($n_j^{(0)}$) denotes the (equilibrium) number density of $j=1,2,3,4$. The interaction rate $\Gamma_1 = n_2 \langle\sigma v\rangle$ depends on the thermally averaged cross-section $\langle\sigma v\rangle$ that we get from quantum field theory (QFT). We provide a qualitative discussion of this equation below.

If $\Gamma_1 \gg \Hp\eu^{-x}$ (recall: $\Hp(x)\eu^{-x}= H$, the expansion rate of the universe), the rate at which the interaction is happening is large enough to keep up with the expansion of the surroundings, and the system is driven towards equilibrium ($n_j = n_j^{(0)}$). When the interactions are sufficiently efficient, we will have equilibrium and the bracket in \cref{mil2:theo:eq:Boltzmann} goes to zero. This is how the Saha equation arises. One may wonder why we consider this approximation at all when we have a completely fine ODE. However, in numerical analysis, the subtraction in said bracket may result in a ``catastrophic cancellation'', hence the Saha approximation for the equilibrium case is necessary.

Once $\Gamma_1 \sim \Hp\eu^{-x}$, the interaction rate is comparable to the expansion rate and this dynamic equilibrium is no longer upheld. We say that the particles involved \textit{decouple} from the primordial plasma. After decoupling, a species evolves independently.

When $\Gamma_1 \ll \Hp\eu^{-x}$, the interaction is irrelevant and the r.h.s. of \cref{mil2:theo:eq:Boltzmann} is practically zero and $n_1$ goes as $\eu^{-3x}$, meaning that the number density of $(1)$ is constant in a comoving volume. If $(1)$ is massive, we say that it \textit{freezes out} at this point, and expect to find the same fractional abundance today. \footnote{\textcolor{blue}{Comment about reionisation?}}


% \feynmandiagram [horizontal=a to b] {
%   i1 [particle=\(e^{-}\)] -- [fermion] a -- [fermion] i2 [particle=\(e^{+}\)],
%   a -- [photon, edge label=\(\gamma\), momentum'=\(k\)] b,
%   f1 [particle=\(\mu^{+}\)] -- [fermion] b -- [fermion] f2 [particle=\(\mu^{-}\)],
% };

\subsubsection{Optical depth and visibility}\label[sec]{mil2:theo:sec:optical_depth}
    Photons travelling through a medium may be absorbed. The intensity of light emitted from a distance $x$ is reduced by the factor $\eu^{-\tau(x)}$ where $\tau(x)$ is the optical depth of the medium. An optically thin ($\tau \ll 1$) medium does little or no absorbing, whereas an optically thick ($\tau \gg 1$) substance does not let much, if any, light pass through. In cosmology, Thomson scattering is predominantly responsible for the absorption of photons universe. This gives us the ODE for $\tau(x)$,
    \begin{equation}\label{mil2:theo:eq:ode_tau_of_x}
        \dv{\tau}{x} = - \frac{cn_e \sigma\ped{T}\eu^{x}}{\Hp(x)},
    \end{equation}
    with $\tau(0)=0$ by definition, where $\sigma\ped{T}$ is the Thomson scattering cross-section and $n_e$ the electron density. A related quantity is the visibility function
    \begin{equation}\label{mil2:theo:eq:gt_of_x}
        \gt(x)= - \eu^{-\tau(x)} \dv{\tau(x)}{x},
    \end{equation}
    a proper probability distribution obeying $\int_{-\infty}^{0}\diff x\, \gt(x) = 1$. The probability it describes, is that of a CMB photon's last interaction with an electron to have happened at $x$. This function will have a peak at the point in time when the mean free path of the photons increased tremendously---at the last scattering surface---which happened immediately after the number of free electrons dropped dramatically. In mathematical terms, this decoupling happened when $\gt(x_*) = \max{\gt(x)}$. 
    
    We will take $x\ped{*}$ to mean the logarithmic cosmic scale factor at the epoch of recombination, and our primary definition of this will be the peak of the visibility function. However, there are ambiguities to the definition of this point in time. For instance, the surface of last scattering can also be taken as the time when the universe is neither opaque nor transparent, i.e.~the solution of $\tau(x\!=\!x\ped{*}) = 1$. As mentioned above, photons decoupled when the interaction rate of the photons was equal the expansion rate of the universe.


    %We will use the former, but expect to see that $\tau(x\ped{*}) \simeq 1$ in any case. 
    %, or
    % \begin{equation}
    %     \dv{\gt(x)}{x}\bigg|_{x=x\ped{dec}} = 0 \quad \Leftrightarrow\quad \left[\dv[2]{\tau(x)}{x} = \left(\dv{\tau(x)}{x} \right)^2 \right]_{x=x\ped{dec}}.
    % \end{equation}
   

    % We will take $x\ped{*}$ to mean the logarithmic scale factor at the time of recombination, which will be so close to the last scattering surface that it will also refer to this point in time when the distinction is not important. 

    

\subsubsection{Hydrogen recombination}\label[sec]{mil2:theo:sec:recombination}
    Before we can compute the optical depth, we need to know the electron number density, $n_e$, at all times. We define the free electron fraction $X_e\equiv n_e/n\ped{b}$ where $n\ped{b}$ is the total baryon number density. Before recombination, that is for $x\ll x_*$, all hydrogen is completely ionised, meaning that $X_e(x\!\ll\!x_*)\simeq 1$. We will consider the 90\% drop in fractional electron abundance mark the onset of recombination, i.e.~when $X_e(x\!=\!x_*) = 0.1$. One should keep in mind that the number 0.1 is arbitrary and that this definition vary in the literature. 

    Consider the interaction that keeps electrons and protons in equilibrium with photons, i.e.~\cref{mil2:theo:eq:relevant_interaction}. Letting $n_s$ ($n_s^{(0)}$) denote the number density of a species/element $s$ (in equilibrium), the corresponding equilibrium equation is 
    \begin{align}\label{mil2:theo:eq:org_Saha}
        \frac{n_e n_p}{n\ped{\element{H}}}  = \frac{n_e^{(0)} n_p^{(0)}}{n\ped{\element{H}}^{(0)}},
    \end{align}
    known as the \textit{Saha equation}. Likewise, letting $m_s$ refer to the mass of $s$, we can take the number density of neutral hydrogen to be
    \begin{equation}
        n\ped{H}= \left(1-Y_P\right) n\ped{b} \simeq \left(1-Y_P\right) \frac{\Omega\ped{b0}\rho\ped{cr0}}{m\ped{H}\eu^{3x}}\,;\quad \rho\ped{cr0} = \frac{3H_0^2}{8\pi G},
    \end{equation}
    where $Y_P$ denotes the primordial helium mass fraction. We neglect helium s.t. $Y_P = 0$ and assume that all baryons are protons. Further, recognising the neutrality of the universe ensures $n_p=n_e$. Now, $n\ped{b}=n_p + n\ped{\element{H}}$ and
    \begin{equation}
        X_e = \frac{n_e}{n_e + n\ped{\element{H}}} = \frac{n_p}{n_p + n\ped{\element{H}}}.
    \end{equation}

    % \textcolor{blue}{
    % The binding energy of hydrogen is $\epsilon_0  = 13.6\unit{eV}$.\footnote{Binding energy $B\ped{H}=m_e + m_p - m\ped{H}$}
    % }
    % \please

    The evolution of the baryon temperature $T\ped{b}=T\ped{b}(x)$ is non-trivial, and the precision gained from implementing the exact model is dissapointing. It suffices to assume $T\ped{b}\approx T\gped{\textgamma} = \TCMB \eu^{-x}$.~\citep{Callin2006}
    
    % As the binding energy of hydrogen is $\epsilon_0 \simeq 13.6\unit{eV}$,\footnote{Binding energy: $\epsilon_0\equiv B\ped{H}=m_e + m_p - m\ped{H}$.} a natural guess is $x_* \approx $ \textcolor{blue}{send help}

    % \sendhelp

    % as it turns out to evolve very much like the photon temperature $T\gped{\textgamma} $.
    
    Let $\varUpsilon= \epskT(T\ped{b})\equiv \epsilon_0/(k\ped{B} T\ped{b})$, where $\epsilon_0 \simeq 13.6\unit{eV}$ is the binding energy of hydrogen, for notational ease. Multiplying~\cref{mil2:theo:eq:org_Saha} by $n\ped{b}^{-1}$ and inserting expressions for $n_s^{(0)}$, we obtain the more suggestive form of the Saha equation
    \begin{equation}\label{mil2:theo:eq:Saha}
        \frac{X_e^2}{1-X_e}= \frac{1}{n\ped{b}}\left(\frac{m_e k\ped{B} T\ped{b}}{2\pi \hbar^2} \right)^{\sfrac{3}{2}} \eu^{-\epskT}\,; \quad   X_e \lse 1.
    \end{equation}
    The constraints on $X_e$ are that it is a positive number that cannot exceed 1 and the observation that it has to be close to 1. The latter constraint is due to the equilibrium assumption from which the Saha equation is derived: as $X_e$ falls the reaction rate for~\cref{mil2:theo:eq:relevant_interaction} falls and equilibrium is not guaranteed. To proceed, we need to solve the Boltzmann equation. More precisely, Jim Peebles needed to solve the Boltzmann equation, whereas we shall study the product; a first-order named ODE the \textit{Peebles equation}. Said equation reads
    \begin{equation}\label{mil2:theo:eq:Peebles}
        \dv{X_e}{x} = \frac{C_r(T\ped{b})}{\Hp(x)\eu^{-x}} \left[ \beta(T\ped{b}) \left(1-X_e\right) - n\ped{\element{H}} \alpha^{(2)}(T\ped{b})X_e^2 \right],
    \end{equation}
    where the necessary mathematical expressions are the following:
    \begin{subequations}\label{mil2:theo:eq:Peebles_add}
        \begin{align}
            C_r(T\ped{b}) &= \frac{\Lambda_{2\gamma}+ \Lambda_{\alpha}}{\Lambda_{2\gamma} + \Lambda_\alpha + \beta^{(2)}(T\ped{b})} \label{mil2:theo:eq:Peebles_Cr} \\
            \Lambda_{2\gamma} &=8.227\unit{s}^{-1} \label{mil2:theo:eq:Peebles_Lambda_2s1s}  \\
            \Lambda_\alpha &= \frac{\Hp\eu^{-x}}{(8\pi)^2 n_{1s}}\left( \frac{3\epsilon_0}{\hbar c}\right)^3 \label{mil2:theo:eq:Peebles_Lambda_alpha}  \\
            n_{1s} &\approx (1-X_e)n\ped{H}\label{mil2:theo:eq:Peebles_n1s}\\
            \beta^{(2)}(T\ped{b}) &= \beta(T\ped{b})\eu^{\sfrac{3}{4}\epskT} \label{mil2:theo:eq:Peebles_beta2} \\
            \beta (T\ped{b}) &= \alpha^{(2)}(T\ped{b}) \left(\frac{m_e k\ped{B} T\ped{b}}{2\pi \hbar^2} \right)^{\sfrac{3}{2}} \eu^{-\epskT} \label{mil2:theo:eq:Peebles_beta}  \\
            \alpha^{(2)}(T\ped{b}) &= \frac{8 c\sigma\ped{T} }{\sqrt{3\pi}}\sqrt{\epskT} \phi_2(T\ped{b}) \label{mil2:theo:eq:Peebles_alpha2} \\
            \phi_2(T\ped{b}) &\approx 0.448\ln{\epskT} \label{mil2:theo:eq:Peebles_phi2}
        \end{align}
    \end{subequations}
    A detailed description of~\cref{mil2:theo:eq:Peebles_add} is found in~\citet{Peebles1968}. We provide a brief summary of the model.
    
    The decay rates $\Lambda_\alpha$ and $\Lambda_{2\gamma}$ are the rates of the processes $2p\to 1s$ and $2s\to 1s$ mentioned in the beginning of \cref{mil2:sec:theo}, respectively. $\beta^{(2)}$ represents the Lyman alpha production.
    Thus, the correction factor $C_r$ is the ratio between the net decay rate and the combined decay and ionisation rates from the first excited level. In other terms, $C_r$ is the probability that a hydrogen atom in $n=2$ actually reaches $n=1$ through emission of a Lyman-\textalpha~photon or two photons. The approximation of the number density of \element{H} in the ground state $n_{1s}$ is sensible for $T\ped{b}\ll 10^5\unit{K}$, i.e.~for $x \ll \ln{\TCMB} -5\ln{10} \approx -10 $. Finally, $\beta$ is the ionization rate and $\alpha^{(2)}$ the recombination rate, where $\phi_2$ is also a good approximation at low temperatures.~\citep{DodelsonBook,ChungPei1995,Peebles1968}

    


\subsubsection{Sound horizon}\label[sec]{mil2:theo:sec:sound_horizon}
    The distance that a sound wave could propagate in the primordial plasma before photons decoupled is called ``the sound horizon at decoupling'', a quantity whose significance will become prominent in sections to come. We define the sound speed of a photon-baryon plasma as
    \begin{equation}
        c_s \equiv c\sqrt{\frac{1}{3(1+R)}},
    \end{equation}
    where the baryon-to-photon energy ratio is defined as
    \begin{equation}\label{mil2:theo:eq:R_of_x}
        R \equiv \frac{3\Omega\ped{b}(x)}{4\Omega\gped{\textgamma}(x)} =  \frac{3\Omega\ped{b0}}{4\Omega\gped{\textgamma 0}}\eu^x.
    \end{equation}
    The comoving distance travelled by a sound wave---the sound horizon---at time $x$ as the solution $s(x)$ to the ODE
    \begin{equation}\label{mil2:theo:eq:ode_rs_of_x}
        \dv{s}{x} = \frac{c_s}{\Hp(x)}\, ; \quad s(x\ped{init}) = \frac{c_s(x\ped{init})}{\Hp(x\ped{init}) }.
    \end{equation}
    Evaluating $\sh\equiv s(x\!=\!x\ped{*})$ gives the sound horizon at decoupling. At this point, the plasma through which sound waves propagate is no longer present and the waves are frozen in.
