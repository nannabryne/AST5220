% !TEX root = ../../main.tex

% ---------------------------------------
% labels: \label{mil2:disc:[type]:[name]}
% ---------------------------------------
% PRESENT TENSE



Overall, the results presented in the figures in~\cref{mil2:sec:res} resemble those of a corresponding analysis by~\citet[see][Fig.~1,~2]{Callin2006}.


The graphs in~\cref{mil2:res:fig:electron_fraction} demonstrate the invalidity of the Saha solution for later times. The solid graph indicates a freeze-out electron fraction of order $10^{-4}$ -- $10^{-3}$, which we confirm in~\cref{mil2:res:tab:time_of_events}. We notice the decelerating decay rate at $x\sim -6.8$. Using our definitions of recombination onset and surface of last scattering, the Saha equation does not suffice for pinpointing the time for when these events occur. In this case, according to~\cref{mil2:res:tab:time_of_events}, the order of these events are reversed. However, these numbers rely on how we define the distinct events.

The optical depth in~\cref{mil2:res:fig:optical_depth} tells us that the universe was very opaque before recombination, and that during the short period around recombination, the optical depth went from $\tau(-7.5)\sim 100$ to $\tau( -6.5)\sim 0.01$. After this epoch, the optical depth resumes its exponential decay. We see that $\tau(x_*)\approx 1$, as expected. 


The visibility function is completely flat (zero) until just before $x=x_*$ where it rapidly increases to its maximum before an almost equally rapid decrease. However, there is a notable asymmetry in the probability function. The tail on the right side of the peak is longer than the one of the left. Together with~\cref{mil2:res:fig:electron_fraction}, we take this to mean that, some time after recombination (see the slope in $X_e(x)$ at $x\sim -6.8$), there are still a significant portion of free electrons left to prevent photons to travel freely. In any case, the CMB photons last interacted with electrons between $x=-7.2$ and $x=-6$, most likely before $x=-6.7$.

\cref{mil2:res:tab:time_of_events} shows that recombination and photon decoupling occurred at redshift 1080 or 380\,000 years after BB, consistent with concordance cosmology (e.g.~\citet[Tab.~3.1]{Baumann}).\footnote{There is an implicit wiggle room when we say ``consistent'' in this context, as there are several ways of defining recombination onset and the last scattering surface.} From this model, we get that the free electron abundance has ceased to be only a single free electron for every 5000 hydrogen atom, i.e.~$X_e(x\!=\!0) = X_e\ap{(fo)}\approx 2\cross 10^{-4}$. \footnote{\textcolor{blue}{Comment about reionisation!}}

Sound waves managed to travel maximum $145\unit{Mpc}$ before frozen in. This length scale is imprinted in the CMB. 