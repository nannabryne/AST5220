% ---------------------------------------
% labels: \label{mil3:disc:[type]:[name]}
% ---------------------------------------
% PRESENT TENSE





The graphs in \cref{mil2:res:fig:electron_fraction} demonstrate the invalidity of the Saha solution for later times and indicates a freeze-out electron fraction of order $10^{-4}$ -- $10^{-3}$. Using our definitions of recombination onset and surface of last scattering, the Saha equation does not suffice for pinpointing the time for when these events occur. According to \cref{mil2:res:tab:time_of_events}, the order of these events are reversed. 

The optical depth and its derivatives are decreasing at a more or less constant rate at all times, except for $-8 < x < -6$, that is around the epoch of recombination $x\sim x_*\approx 7$. In this period of time, the universe goes from being very opaque, with $\tau(\sim-8)\sim 100$, to becoming transparent, with $\tau(\sim -6)\sim 0.01$. We see that $\tau(x_*)\simeq 1$, as expected. 

The visibility function is completely flat (zero) until just before $x_*$ where it rapidly increases to its maximum before an almost equally rapid decrease. The tail on the right side of the peak is longer than the one of the left, meaning that there are free electrons preventing photons to travel freely
 % consisten with what we see in fig 6

% initially sudden disappearance of a large portion of the electrons is slowing down and still preventin
In any case, the CMB photons last interacted with electrons between $x=-7.2$ and $x=-6$, most likely before $x=-6.5$.

\colorbox{pink}{\dots \, \dots \, \dots \, \dots \, \dots \, \dots \, \dots }


\cref{mil2:res:tab:time_of_events} shows that recombination/photon decoupling occurred at redshift 1080 or 380\,000 years after BB, consistent with concordance cosmology (e.g. \citep[Tab. 3.1]{Baumann}).\footnote{There is an implicit wiggle room when we say ``consistent'' in this context, as there are several ways of defining recombination onset and the last scattering surface.} 