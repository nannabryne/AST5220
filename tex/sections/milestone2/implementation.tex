% !TEX root = ../../main.tex

% --------------------------------------
% labels: \label{mil2:imp:[type]:[name]}
% --------------------------------------
% PAST TENSE

We implemented a class in \verb|C++| that assumes a background, an object of the class from~\cref{mil1:sec:imp}, and a primordial helium abundance $Y_P$. The code we wrote calculated $X_e(x)$ (and $n_e(x)$) from the Saha and Peebles equations and computed $\tau(x)$ as well as $\gti(x)$, $\dv*[]{\gti(x)}{x}$ and $\dv*[2]{\gti(x)}{x}$ for a given array of $x$-values. We added a computation of $s(x)$ to find the sound horizon at decoupling.

We ran an additional simulation where we assumed equilibrium all the way, that is we solved for $X_e(x)$ using only the Saha equation. 

\subsubsection{Electron fraction and number density}\label[sec]{mil2:imp:sec:Xe_ne}
    The computation of $X_e(x)$ was divided into two parts. We let $X_e(x)>0.99$ signify the Saha regime for which we use the Saha approximation in~\cref{mil2:theo:eq:Saha}. Numerically speaking, the solution to this equation, 
    \begin{equation}\label{mil2:imp:eq:sol_Saha}
        X_e(x) = \frac{\expr}{2}\left( -1+\sqrt{1+4\expr^{-1}}\right)\,; \quad 
            \expr = \frac{1}{n\ped{b}} \left( \frac{m_e k\ped{B} T\ped{b} }{2\pi \hbar^2} \right)^{\sfrac{3}{2}} \eu^{\epskT},
    \end{equation}
    blows up for large values of $\expr$. Analytically we have $X_e\to 1$ as $\expr\to \infty$, so we circumvented the issue by letting $X_e=1 \,\forall\, \expr > 10^7$. Outside the Saha regime, we used the last value of $X_e(x)$ ($\approx 0.99$) as initial condition on the Peebles ODE (\cref{mil2:theo:eq:Peebles}) and solved this for the remaining values of $x$. Here as well, we needed to address numerical stability concerns. Specifically, $\beta^{(2)}$ in~\cref{mil2:theo:eq:Peebles_beta2} was set to zero when the exponent was too large; $\epskT> 200$, i.e. at later times when $T\ped{b} < 0.005 \epsilon_0/k\ped{B} \sim 800$~K. 

    We solved for $8\cross10^5+1$ equally spaced points $x\in[-12, 0]$. The numerical integration was performed using RK4 and the number of steps left in the $x$-array at the end of the Saha regime. 


\subsubsection{Optical depth and visibility}\label[sec]{mil2:imp:sec:tau_gt}
    We computed $\tau(x)$ by solving the simple ODE in~\cref{mil2:theo:eq:ode_tau_of_x} starting from $x=0$ where $\tau(0)=0$ and integrating backwards in time, using $8\cross10^5$ steps along $x\in[0, -12]$ in the RK4 algorithm.
    
    We used the analytical expressions in~\cref{mil2:theo:eq:ode_tau_of_x} and~\cref{mil2:theo:eq:gt_of_x} for $\dv*{\tau(x)}{x}$ and $\gti(x)$, respectively, and found $\dv*[2]{\tau(x)}{x}$, $\dv*{\gti(x)}{x}$ and $\dv*[2]{\gti(x)}{x}$ numerically. 

\subsubsection{Sound horizon}\label[sec]{mil2:imp:sec:rs}
    The computation of $s(x)$ was a straight-forward numerical integration of~\cref{mil2:theo:eq:ode_rs_of_x} for $x\in[-12, 0]$, again using RK4 with $8\cross10^5$ steps. Having $x\ped{init}=-12$ ensures radiation domination (see \cref{mil1:res:fig:density_params}), so the initial condition is valid.