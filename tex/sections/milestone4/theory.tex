% !TEX root = ../../main.tex

% ---------------------------------------
% labels: \label{mil4:theo:[type]:[name]}
% ---------------------------------------
% PRESENT/FUTURE TENSE




\subsubsection{From quantum to classical field theory}
    \dots


    Let $\xi(\vec{x}) \equiv \xi(\vec{y}-\vec{z})$ denote the two-point correlation function for the quantum operator $\hat{f}$, i.e.
    \begin{equation}
        \xi(\vec{x}) = \avg{ 0 | f^\dagger(t,\vec{z}) f(t, \vec{y}) |  0}.
    \end{equation}
    We can define the dimensionless power spectrum $\Delta_f^2(t, k)$ by
    \begin{equation}
        \xi = \int \dx{\ln{k}} \abs{\Delta_f(t,k)}^2 P_\mathcal{R}(k)
    \end{equation}


\please

The angular CMB power spectrum is given by
\begin{equation}
    C(\ell)= \frac{2}{\pi} \int \frac{d^3 k}{(2\pi)^2} P_P(k) \mathcal{T}_\ell^2 (k),
\end{equation}
where
\begin{equation}
    \mathcal{T}_\ell(k) \equiv \frac{\Theta_\ell(x\!=\!0, k)}{\mathcal{R(\vec{k})}}
\end{equation}
is the transfer function and $P_P(k)$ is the primordial power spectrum given by
\begin{equation}
    \frac{k^3}{2\pi^2} P_P (k) = A\ped{S}\left( \frac{k}{H_0}\right)^{n\ped{S}-1},
\end{equation}
$n\ped{S}$ being the spectral index (scalar perturbation).


Define $\mathcal{D}_\ell \equiv \ell (\ell+1)/(2\pi) \cdot C(\ell) $.

\sendhelp

\please

The angular CMB power spectrum is given by
\begin{equation}
    C(\ell)= \frac{2}{\pi} \int_0^\infty \dx{k} k^2 P_\mathcal{R}(k) \abs{\mathcal{T}_\ell (k)}^2,
\end{equation}
where
\begin{equation}
    \mathcal{T}_\ell(k) \equiv \frac{\Theta_\ell(x_0, k)}{\mathcal{R(\vec{k})}}
\end{equation}
is the transfer function and $P_\mathcal{R}(k)$ is the primordial power spectrum given by
\begin{equation}
    \frac{k^3}{2\pi^2} P_\mathcal{R} (k) = \mathcal{A}\ped{S}\left( \frac{k}{k\ped{p}}\right)^{n\ped{S}-1},
\end{equation}
$n\ped{S}$ being the spectral index and $\mathcal{A}\ped{S}$ the primordial amplitude.


Define $\mathcal{D}_\ell \equiv \ell (\ell+1)(2\pi)^{-1} C(\ell) $.


% Using~\cref{mil3:theo:eq:temp_multipole_and_source_func}, we find that 
% \begin{equation}
%     C(\ell) = 
% \end{equation}


\sendhelp


\subsubsection{power spectra, priomordial blah blah}
    We revisit the curvature perturbation $\mathcal{R}(\vec{k})$ immediately after inflation, that which determines the initial conditions for each Fourier mode. During the generation of fluctuations, there should be no preferred position nor direction in space, resulting in statistical homogeneity and isotropy of said fluctuations. We let $\avg{\cdot}$ denote the statistical expectation value over many realisations of the universe. Thus, the primordial power spectrum $P_\mathcal{R}(k)$ is given by
    \begin{equation}
        k^3 \avg{\mathcal{R}(\vec{k})\mathcal{R}^*(\vec{k'})} = 2\pi^2P_\mathcal{R}(k).
    \end{equation}
    Further, we assume fluctuations to be gaussian. The perturbations are adiabatic and their averages vanish. Now, the power spectrum of perturbation variable $\hat{f}$ is the convolution \textcolor{blue}{(check this)} between initial power spectrum and the deterministic transfer function $f^2$:
    \begin{equation}
    \begin{split}
        k^3\avg{ \hat{f}(t,\vec{k}) \hat{f}^\ast (t,\vec{k'})} &= 2\pi^2 \deltafunc{\vec{k}-\vec{k'}} P_f (x,k) \\
        P_f(x,k) &= f^2(x,k) P_\mathcal{R}(k)
    \end{split}
    \end{equation}



    Recall: $d^3k = k^2 \dx{k} \dOmg[k]$


\subsubsection{SOMETHING TITLE PHOTON I GUESS}
\textcolor{blue}{blah blah blah intro something}


We begin by expanding the temperature perturbation $\Theta(t, \vec{x}, \vec{\hat{p}})$ into spherical harmonics, i.e.
\begin{equation}
    \Theta(t, \dots) = \sum_{\ell=1}^{\infty} \sum_{m=-\ell}^{\ell} a_{\ell m}(t, \vec{x}) Y_{\ell m} (\vec{\hat{p}})
\end{equation}
The orthogonality of the spherical harmonics $Y_{\ell m}$\footnote{$\int \dOmg[n] Y\ellm(\vec{\hat{n}}) Y\ellmp^*(\vec{\hat{n}}) = \deltasym{\ell \ell'} \deltasym{m m'} $.} gives the coefficients
\begin{equation}
    a\ellm = a\ellm(t,\vec{x}) = \int \dOmg[p] Y^*_{\ell m} (\vec{\hat{p}}) \Theta(t, \vec{x}, \vec{\hat{p}}),
\end{equation}
where $d\varOmega_{\vec{\hat{p}}} = \sin{\vartheta} \diff \vartheta\diff \varphi$ is the infinitesimal solid angle element of $\vec{\hat{p}} = (\vartheta, \varphi)$. Further expanding into Fourier modes and multipoles:
\begin{equation}
\begin{split}
    a_{\ell m} &= \int \dOmg[p] Y^*_{\ell m} (\vec{\hat{p}}) \int \frac{d^3 k}{(2\pi)^3} \eu[\im \vec{k}\cdot \vec{x}] \Theta(x, \vec{k}, \vec{\hat{p}})  \\
    &= \int \dOmg[p] Y^*_{\ell m} (\vec{\hat{p}}) \int \frac{d^3 k}{(2\pi)^3} \eu[\im \vec{k}\cdot \vec{x}]\sum_{\ell} \frac{2\ell + 1}{\im^\ell} \mathcal{P}_\ell(\mu) \Theta_\ell(x, \vec{k})
\end{split}
\end{equation}
In the second line, we used that $\mu=\vec{\hat{p}}\cdot\vec{k}/k$ to let $\Theta(x,\vec{k}, \vec{\hat{p}}) \to\Theta(x,\vec{k},\mu) $.


To calculate the \textit{angular} correlation function $\langle a\ellm a^*\ellmp \rangle$, we need the \textcolor{blue}{two-point correlation function $\langle \Theta() \Theta^*()\rangle$} 



We reintroduce the curvature perturbation $\mathcal{R}(\vec{k})$ by letting $\Theta_\ell(x,\vec{k}) \to \Theta_\ell(x,k) \mathcal{R}(\vec{k})$ in the equations above, where $\Theta_\ell(x, k)$ are the perturbations we computed in~\cref{sec:mil3}. To find the ensamble average $\langle a\ellm a^*\ellmp \rangle = C(\ell) \deltasym{\ell \ell'} \deltasym{mm'}$, we can factor out everything except for that which depends on the initial amplitude of a mode. Now, with our substitution, we get the only relevant average distrubution:
%let $\langle \dot \rangle$ denote the average o
\begin{equation}
\begin{split}
    \langle \Theta_\ell(x,\vec{k}) \Theta_\ell^*(x,\vec{k'})\rangle \to& \Theta_\ell(x,k) \Theta_\ell^*(x,k') \langle \mathcal{R}(\vec{k}) \mathcal{R}^*(\vec{k'}) \rangle \\
    &=  \Theta_\ell(x,k) \Theta_\ell^*(x,k')(2\pi)^3 \deltafunc{\vec{k}-\vec{k'}} P_\mathcal{R}(k) \\
    &= (2\pi)^3 \abs{\Theta_\ell(x,k)}^2 P_\mathcal{R}(k) 
\end{split}
\end{equation}
We can write the monster as
\begin{equation}
\begin{split}
    \langle a\ellm a^*\ellmp \rangle =& \sum_{\ell_1 \ell_2} (2\ell_1 +1) (2\ell_2 +1) \int \frac{d^3k}{(2\pi)^3} \int \frac{d^3k'}{(2\pi)^3}\eu[\im (\vec{k}-\vec{k'})\cdot \vec{x}] \\
    &\int \dOmg[p] \int \dOmg[p'] Y_{\ell m}^*(\vec{\hat{p}}) Y_{\ell' m'}(\vec{\hat{p}'}) \mathcal{P}_{\ell_1}(\mu)\mathcal{P}_{\ell_2}(\mu') P_\mathcal{R}(\vec{k})
\end{split}
\end{equation}

% \begin{equation}
% \begin{split}
%     a\ellm a\ellmp = 
% \end{split}
% \end{equation}





\subsubsection{Source function---line-of-sight integration}
    The temperature source function $\St(x,k)$ can be written as the sum of four terms,
    \begin{equation}\label{mil4:theo:eq:source_function_four_terms}
        \St(x,k) = \text{(SW)} + \text{(ISW)} + \text{(Doppler)} + \text{(quadrupole)}.
    \end{equation}
    The first term---the Sachs-Wolfe term---is the main contribution to the source function and is given by
    \begin{equation}
        \text{(SW)} = \gt(x) \left[\Theta_0(x,k) + \Psi(x,k) + \frac{1}{4}\Theta_2(x,k) \right].
    \end{equation}
    The term originates in the gravitational redshifting of CMB photons as they propagate through the gravitational field.

    The Integrated Sachs-Wolfe effect %shows up in the second term in~\cref{mil4:theo:eq:source_function_four_terms}
    \begin{equation}
        \text{(ISW)} = \eu[-\tau(x)] \left[ \dv{\Psi(x,k)}{x}-  \dv{\Phi(x,k)}{x} \right]
    \end{equation}
    arises from the time-dependent nature of the gravitational potential.

    The Doppler effect due to our galaxy's motion relative to the CMB rest frame contributes to the source function in the form of
    \begin{equation}
        \text{(Doppler)} = - \frac{1}{ck} \dv{}{x} \left[\Hp(x) \gt(x) u\ped{b}(x,k) \right].
    \end{equation}

    The smallest contribution is the quadrupole term 
    \begin{equation}
        \text{(quadrupole)} =  \frac{3}{4c^2k^2} \dv{}{x} \left(\Hp(x) \dv{}{x} \left[ \Hp(x) \gt(x) \Theta_2(x,k) \right] \right).
    \end{equation}



\subsubsection{Matter power spectrum}
    \textcolor{blue}{blah blah blah something something}
    The dimensionless function \dots
    \begin{equation}\label{mil4:theo:eq:Delta_s_of_x_k}
        \Delta_s (x,k) = \delta_s(x,k) - \frac{3(1+w_s)\Hp(x)}{ck}u_s(x,k)
    \end{equation}

    \begin{equation}\label{mil4:theo:eq:matter_power_spectrum}
        P(x,k) = \abs{\Delta\ped{m}(x,k)}^2 P_\mathcal{R}(k)
    \end{equation}


    