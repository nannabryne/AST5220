% !TEX root = ../../main.tex

% ---------------------------------------
% labels: \label{mil4:theo:[type]:[name]}
% ---------------------------------------
% PRESENT/FUTURE TENSE




The power spectrum, denoted $P(\vec{k})$, is defined as the Fourier transform of the two-point correlation function, denoted $\xi(\vec{x})$. That is
\begin{equation}
    \xi(\vec{x}) = \int \dkkk  \eu[\im \vec{k}\cdot \vec{x}] P(\vec{k}),
\end{equation}
and by performing a Henkel transform we get
\begin{equation}
    \xi(r) = \int\dx{k}\frac{k^2}{2\pi^2} \frac{\sin(kr)}{kr} P(k) = \int\!\!\frac{\diff k}{k} \frac{\sin(kr)}{kr} \frac{k^3P(k)}{2\pi^2},
\end{equation}
where $r=\abs{\vec{x}}$.


\paragraph{Notational note} 
To not be overwhelmed by notation in this section, we let $t$ be the time variable in real space and $x$ the time variable in Fourier space, such that in general,
\begin{equation}
    f(t, \vec{x}) = \int \dkkk f(x, \vec{k}).
\end{equation}
We will also account for the primordial fluctuations by setting $f(x,\vec{k})= f(x, k) \mathcal{R}(\vec{k})$, where $f(x,k)$ is a computed perturbation quantity from~\cref{sec:mil3}. To be very clear, $x\neq \abs{\vec{x}}$.






% \subsubsection{From quantum to classical field theory}
%     \dots


%     Let $\xi(\vec{x}) \equiv \xi(\vec{y}-\vec{z})$ denote the two-point correlation function for the quantum operator $\hat{f}$, i.e.
%     \begin{equation}
%         \xi(\vec{x}) = \avg{ 0 | f^\dagger(t,\vec{z}) f(t, \vec{y}) |  0}.
%     \end{equation}
%     We can define the dimensionless power spectrum $\Delta_f^2(t, k)$ by
%     \begin{equation}
%         \xi = \int \dx{\ln{k}} \abs{\Delta_f(t,k)}^2 P_\mathcal{R}(k)
%     \end{equation}


% \please

% The angular CMB power spectrum is given by
% \begin{equation}
%     C(\ell)= \frac{2}{\pi} \int \frac{d^3 k}{(2\pi)^2} P_P(k) \mathcal{T}_\ell^2 (k),
% \end{equation}
% where
% \begin{equation}
%     \mathcal{T}_\ell(k) \equiv \frac{\Theta_\ell(x\!=\!0, k)}{\mathcal{R(\vec{k})}}
% \end{equation}
% is the transfer function and $P_P(k)$ is the primordial power spectrum given by
% \begin{equation}
%     \frac{k^3}{2\pi^2} P_P (k) = A\ped{S}\left( \frac{k}{H_0}\right)^{\ns-1},
% \end{equation}
% $\ns$ being the spectral index (scalar perturbation).


% Define $\mathcal{D}_\ell \equiv \ell (\ell+1)/(2\pi) \cdot C(\ell) $.

% \sendhelp

% \please

% The angular CMB power spectrum is given by
% \begin{equation}
%     C(\ell)= \frac{2}{\pi} \int_0^\infty \dx{k} k^2 P_\mathcal{R}(k) \abs{\mathcal{T}_\ell (k)}^2,
% \end{equation}
% where
% \begin{equation}
%     \mathcal{T}_\ell(k) \equiv \frac{\Theta_\ell(x_0, k)}{\mathcal{R(\vec{k})}}
% \end{equation}
% is the transfer function and $P_\mathcal{R}(k)$ is the primordial power spectrum given by
% \begin{equation}
%     \frac{k^3}{2\pi^2} P_\mathcal{R} (k) = \As\left( \frac{k}{k\ped{p}}\right)^{\ns-1},
% \end{equation}
% $\ns$ being the spectral index and $\As$ the primordial amplitude.


% Define $\mathcal{D}_\ell \equiv \ell (\ell+1)(2\pi)^{-1} C(\ell) $.


% Using~\cref{mil3:theo:eq:temp_multipole_and_source_func}, we find that 
% \begin{equation}
%     C(\ell) = 
% \end{equation}


% \sendhelp

\paragraph{Power spectra}
We revisit the curvature perturbation $\mathcal{R}(\vec{k})$ immediately after inflation, that which determines the initial conditions for each Fourier mode. During the generation of fluctuations, there should be no preferred position nor direction in space, resulting in statistical homogeneity and isotropy of said fluctuations. We let $\avg{\cdot}$ denote the statistical expectation value over many realisations of the universe. Thus, the primordial power spectrum $P_\mathcal{R}(k)$ is given by
\begin{equation}\label{mil4:theo:eq:primordial_power_spectum}
    \avg{\mathcal{R}(\vec{k})\mathcal{R}^*(\vec{k'})} = (2\pi)^3 \deltafunc{\vec{k}-\vec{k'}}P_\mathcal{R}(k).
\end{equation}
The inflationary model we consider predicts the dimensionless primordial power spectrum
\begin{equation}\label{mil4:theo:eq:Delta_R_of_k}
    \Delta^2_\mathcal{R} (k) = \frac{k^3}{2\pi^2} P_\mathcal{R} (k)= \As \left( \frac{k}{k\ped{p}} \right)^{\ns -1},
\end{equation}
where $\As$ is the primordial amplitude, $\ns$ the scaler spectral index and $k\ped{p}$ a fundamental pivot scale.

Further, we assume fluctuations to be gaussian. The perturbations are adiabatic and their averages vanish. Now, the power spectrum of perturbation variable $f$ is the combination between the initial power spectrum and a deterministic transfer function $f^2$:
\begin{equation}\label{mil4:theo:eq:defining_power_spectrum}
\begin{split}
    \avg{f(x,\vec{k}) f^\ast(x,\vec{k'})} &= f(x,k)f^\ast(x,k') \avg{\mathcal{R}(\vec{k})\mathcal{R}^*(\vec{k'})} \\
    &= (2\pi)^3 \deltafunc{\vec{k}-\vec{k'}} \abs{f(x,k)}^2 P_\mathcal{R}(k)
\end{split}
\end{equation}


For a field on a 3D sphere $g(\vec{\hat{n}})$, we can expand the directional dependence in spherical harmonics,
\begin{equation}\label{mil4:theo:eq:spherical_harmonics_expansion}
    g(\vec{\hat{n}}) = g(t, \vec{x}, \vec{\hat{n}})=  \sum_{\ell} \sum_{m=-\ell}^{\ell} a_{\ell m}(t, \vec{x}) Y_{\ell m} (\vec{\hat{n}}),
\end{equation}
and the (angular) power spectrum $C(\ell)$ today is given by
\begin{equation}\label{mil4:theo:eq:defining_angular_power_spectrum}
    \avg{a\ellm(t_0, \vec{x}) a^\ast\ellmp(t_0, \vec{x})}= \deltasym{\ell \ell'} \deltasym{m m'} C(\ell).
\end{equation}



\subsubsection{The CMB power spectrum}

We begin by expanding the temperature perturbation $\Theta(t, \vec{x}, \vec{\hat{p}})$ into spherical harmonics, as in~\cref{mil4:theo:eq:spherical_harmonics_expansion}.
% \begin{equation}
%     \Theta(t, \vec{x}, \vec{\hat{p}}) = \sum_{\ell} \sum_{m=-\ell}^{\ell} a_{\ell m}(t, \vec{x}) Y_{\ell m} (\vec{\hat{p}}),
% \end{equation}
% where
% \begin{equation}
%     \avg{a\ellm a\ellmp^*} = \deltasym{\ell \ell'} \deltasym{m m'} C(\ell)
% \end{equation}
% defines the power spectrum $C(\ell)$ in this context. 
The orthogonality of $Y_{\ell m}(\vec{\hat{p}})$\footnote{$\int \dOmg[n] Y\ellm(\vec{\hat{n}}) Y\ellmp^*(\vec{\hat{n}}) = \deltasym{\ell \ell'} \deltasym{m m'} $.} gives the coefficients
\begin{equation}
    a\ellm = a\ellm(t,\vec{x}) = \int \dOmg[p] Y^*_{\ell m} (\vec{\hat{p}}) \Theta(t, \vec{x}, \vec{\hat{p}}),
\end{equation}
where $d\varOmega_{\vec{\hat{p}}} = \sin{\vartheta} \diff \vartheta\diff \varphi$ is the infinitesimal solid angle element of $\vec{\hat{p}} = (\vartheta, \varphi)$. Further, we expand the coefficients into Fourier modes and multipoles:
\begin{equation}\label{mil4:theo:eq:a_ellm_expansions}
\begin{split}
    a_{\ell m} &= \int \dOmg[p] Y^*_{\ell m} (\vec{\hat{p}}) \int \dkkk \eu[\im \vec{k}\cdot \vec{x}] \Theta(x, \vec{k}, \vec{\hat{p}})  \\
    &= \int \dOmg[p] Y^*_{\ell m} (\vec{\hat{p}}) \int \dkkk \eu[\im \vec{k}\cdot \vec{x}]\sum_{\ell'} \frac{2\ell' + 1}{\im^{\ell'}} \mathcal{P}_{\ell'}(\mu) \Theta_{\ell'}(x, \vec{k})
\end{split}
\end{equation}
In the second line, we used that $\mu=\vec{\hat{p}}\cdot\vec{\hat{k}}$ ($\vec{\hat{k}}\equiv \vec{k}/k$) to let $\Theta(x,\vec{k}, \vec{\hat{p}}) \to\Theta(x,\vec{k},\mu) $. Taking the ensemble average $\langle a\ellm a^*\ellmp \rangle$ will only affect the $\Theta_{\ell'}(x, \vec{k})$'s in the expression above, as this depends on the initial amplitude of a mode. As initially explained, we reintroduce the curvature perturbation $\mathcal{R}(\vec{k})$ by substituting $\Theta_\ell(x,\vec{k}) \to \Theta_\ell(x,k) \mathcal{R}(\vec{k})$, where $\Theta_\ell(x, k)$ are the perturbations we computed in~\cref{sec:mil3}. We get the only relevant average distribution:

\begin{equation}\label{mil4:theo:eq:substitution_idk}
\begin{split}
    \langle \Theta_\ell(x,\vec{k}) \Theta_{\ell'}^*(x,\vec{k'})\rangle \to \Theta_\ell(x,k) \Theta_{\ell'}^*(x,k') \langle \mathcal{R}(\vec{k}) \mathcal{R}^*(\vec{k'}) \rangle \,; \\
    \quad \langle \mathcal{R}(\vec{k}) \mathcal{R}^*(\vec{k'}) \rangle = (2\pi)^3 \deltafunc{\vec{k}-\vec{k'}} P_\mathcal{R}(k) 
    % &=  \Theta_\ell(x,k) \Theta_\ell^*(x,k')(2\pi)^3 \deltafunc{\vec{k}-\vec{k'}} \Delta^2_\mathcal{R}(k) %\\
    % &= (2\pi)^3 \abs{\Theta_\ell(x,k)}^2 \Delta ^2_\mathcal{R}(k) 
\end{split}
\end{equation}
Writing it out, we get
\begin{equation}\label{mil4:theo:full_expr_idk}
\begin{split}
    \langle a\ellm a^*\ellmp \rangle =&  \int \dOmg[p] \int \dOmg[p'] Y_{\ell m}^*(\vec{\hat{p}}) Y_{\ell' m'}(\vec{\hat{p}'})  \\
    &\cdot \int \dkkk P_\mathcal{R}(k)  \sum_{\ell_1 \ell_2} (2\ell_1 +1) (2\ell_2 +1) (-\im)^{\ell_1-\ell_2} \\
    &\cdot \mathcal{P}_{\ell_1}(\mu)\mathcal{P}_{\ell_2}(\mu')\Theta_{\ell_1}(x,k) \Theta^{*}_{\ell_2}(x,k) ,
\end{split}
\end{equation} 
where we used the Dirac-Delta function from~\cref{mil4:theo:eq:substitution_idk} to eliminate $\int \! d^3k' (2\pi)^{-3} \eu[\im (\vec{k}-\vec{k'})\cdot \vec{x}]$. The identity 
\begin{equation}
    \int \dOmg[p] Y\ellm (\vec{\hat{p}}) \mathcal{P}_{\ell'}(\vec{\hat{k}}\cdot \vec{\hat{p}}) = \deltasym{\ell \ell'} \frac{4\pi}{2\ell +1}Y\ellm (\vec{\hat{k}})
\end{equation}
gives rise to factors $\deltasym{\ell \ell_1}$ and $\deltasym{\ell' \ell_2}$, killing the infinite sums in~\cref{mil4:theo:full_expr_idk}. We are left with
\begin{equation}
    \langle a\ellm a^*\ellmp \rangle =  (4\pi)^2 \int \dkkk Y_{\ell m}^*(\vec{\hat{k}}) Y_{\ell' m'}(\vec{\hat{k}})  \Theta_{\ell}(x,k) \Theta^{*}_{\ell'}(x,k) P_\mathcal{R}(k).
\end{equation}
Spherical symmetry ($d^3k \to k^2 \, \dx{k} \,\dOmg[k]$) gives 
\begin{equation}
    \langle a\ellm a^*\ellmp \rangle = \frac{2}{\pi} \int \dOmg[k] Y_{\ell m}^\ast(\vec{\hat{k}}) Y_{\ell' m'}(\vec{\hat{k}}) \int \dx{k} k^2  \Theta_{\ell}(x,k) \Theta^{*}_{\ell'}(x,k)  P_\mathcal{R}(k),
\end{equation}
and we use the orthogonality relation of the spherical harmonics to find
\begin{equation}
    \langle a\ellm a^*\ellmp \rangle = \deltasym{\ell \ell'} \deltasym{m m'} \cross \frac{2}{\pi}\int \dx{k} k^2  \abs{\Theta_{\ell}(x,k)}^2P_\mathcal{R}(k).
\end{equation}
Evaluated at $x=x_0=0$, we find the expression for the angular CMB power spectrum today (see~\cref{mil4:theo:eq:defining_angular_power_spectrum}). We can rewrite this in terms of the dimensionless primordial power spectrum $\Delta^2_\mathcal{R} = k^3/(2\pi^2) P_\mathcal{R}(k)$. Now, the angular CMB power spectrum is given by
\begin{equation}\label{mil4:theo:eq:C_of_ell_final}
    C(\ell) = 4\pi \int_0^{\infty} \dx{k} \frac{\abs{\Theta_\ell(0,k)}^2}{k} \Delta_\mathcal{R}^2(k),
\end{equation}
however, it is customary to present is in a scaled version $\mathcal{D}(\ell) \equiv \ell (\ell +1) (2\pi)^{-1} \TCMB^2 C(\ell)$.

The cosmic variance is an unavoidable error in the estimation of $C(\ell)$, given by 
\begin{equation}\label{mil4:theo:eq:cosmic_variance}
    \left( \frac{\Delta C(\ell)}{C(\ell)}\right)_\text{cosmic variance} = \sqrt{\frac{2}{2\ell +1}}.
\end{equation}
Due to the fact that this is very large for small $\ell$, the sums over multipole moment $\ell$ (in e.g.~\cref{mil4:theo:eq:a_ellm_expansions}) will normally start at $\ell=2$ and stop at some $\ell\ped{MAX}$. The monopole term ($\ell=0$) is simply the average temperature across the CMB sky, whereas the dipole term ($\ell=1$) is affected by our motion relative to the CMB rest frame. 

We usually relate the angular wavenumber $\ell$ to the angular scale $\theta$ by $\theta \sim 180^{\circ} /\ell$. 


\subsubsection{Source function}
    The line-of-sight integral that gives the photon temperature multipoles $\Theta_\ell(0,k)$ can be written
    \begin{equation}\label{mil4:theo:eq:los_integration}
    \begin{split}
        \Theta_\ell (0, k) &= \int_{-\infty}^{0} \dx{x} \St(x,k) j_\ell(k\left[\eta_0 -\eta(x)\right]) \\
        &\simeq  \int_{\lesssim x_*}^{0} \dx{x} \St(x,k) j_\ell(k\left[\eta_0 -\eta(x)\right]),
    \end{split}
    \end{equation}
    since (as we will see) $\St(x,k)$ vanishes before decoupling. The temperature source function $\St(x,k)$ can be written as the sum of four terms,
    \begin{equation}\label{mil4:theo:eq:source_function_four_terms}
        \St(x,k) = \text{(SW)} + \text{(ISW)} + \text{(Doppler)} + \text{(pol)}.
    \end{equation}
    We will break this expression down, term by term. 

    \begin{subequations}\label{mil4:theo:eq:source_terms}
    The first term---the Sachs-Wolfe term---is the predominant contribution to the source function and is given by
    \begin{equation}
        \text{(SW)} = \gt(x) \left[\Theta_0(x,k) + \Psi(x,k) + \frac{1}{4}\Theta_2(x,k) \right].
    \end{equation}
    The term originates in the gravitational redshifting of CMB photons at the surface of last scattering. For the largest scales ($\ell \lesssim 20$), the corresponding contribution to the CMB anisotropy, $C(\ell)^\mathrm{SW}$, to plateau---hence the name ``Sachs-Wolfe plateau''. By a rough approximation, we can get to $\Theta_\ell(0,k) \approx \left( \Theta_0(x_*,k) + \Psi(x_*, k)\right) j_\ell(k\eta_0)$. Leaving the steps out, we take note that the effective temperature perturbation $\left[\Theta_0 + \Psi\right](x\!\leq \! x_*, k)$ can be used to study the basic structures of the CMB power spectrum. We do this in~\cref{mil4:theo:sec:eff_temp_pert}.

    Also due to gravitational redshift is the integrated Sachs-Wolfe effect,
    \begin{equation}
        \text{(ISW)} = \eu[-\tau(x)] \left[ \dv{\Psi(x,k)}{x}-  \dv{\Phi(x,k)}{x} \right].
    \end{equation}
    It occurs between the last scatting surface and us, thus not relevant in the primordial CMB. The phenomenon arises when there is a time dependence in the gravitational potentials. Hence, in the era dominated by normal matter, this effect vanishes (there is little or no evolution to the gravitational wells and hills). We therefore speak of two sides to this effect---before and after matter domination. Following the generation of the primordial CMB, by courtesy of the (non-integrated) SW effect, the ``early-time'' ISW take immediate effect. As with the presence of radiation, dark energy affects the expansion of the universe and disturbs the gravitational potential. We expect to see the ``late-time'' ISW affecting the large-modes that entered (and enters) the horizon after matter-dark energy equality.

    The Doppler effect due to our galaxy's peculiar motion relative to the CMB rest frame contributes to the source function in the form of
    \begin{equation}
        \text{(Doppler)} = - \frac{1}{ck} \dv{}{x} \left[\Hp(x) \gt(x) u\ped{b}(x,k) \right].
    \end{equation}
    The shifted frequency and apparent deflection of the CMB photons distort $a\ellm$ at all $\ell$s.

    The smallest contribution is the quadrupole term 
    \begin{equation}
        \text{(pol)} =  \frac{3}{4c^2k^2} \dv{}{x} \left(\Hp(x) \dv{}{x} \left[ \Hp(x) \gt(x) \Theta_2(x,k) \right] \right),
    \end{equation}
    \end{subequations}
    for which we do not have a statisfying physical intuition. In any case, this term will not provide any new insights in our polarisation-free neutrino-neglecting model.


\subsubsection{Effective temperature perturbation}\label[sec]{mil4:theo:sec:eff_temp_pert}
    We intend to give a high-level asymptotic analysis of the CMB power spectrum to be able to extract physics from it. For more details, see~\citet{WintherIsComing},~\citet{DodelsonBook} and especially~\citet{waynehu}. In this section, we allow ourselves the freedom of using notation unfamiliar to most people,\footnote{Partly because it is made-up.} where the symbols s.a. $\rightsquigarrow, \sim, \dots$ essentially mean ``goes as'' or ``is determined by'', depending on the context or what we want to emphasise. $A\subset \sum \cdot$ means that $A$ is included in the sum either as one of the terms or as the essential part of the term.

    We consider the approximation on the photon multipole
    \begin{equation}\label{mil4:theo:eq:theta_ell_approx}
        \Theta_\ell(0,k)\approx  \Theta_\ell^\text{SW  }(0,k) \approx \left[\Theta_0+\Psi\right](x_*,k)j_\ell(k\chi_*),
    \end{equation}
    where $\chi_* = \eta_0-\eta_*\approx \eta_0$ is the comoving distance at decoupling. The equation for this quantity describes a driven harmonic oscillator, and going back to real space and ignoring the driving force, we obtain an equation for $\delta_\gamma(t, \vec{x})$ describing waves moving at the speed of sound $c_s=c\sqrt{w\gped{\textgamma}}=c/\sqrt{3}$. 

    

    \paragraph{Gravitational forcing}
    Further, it can be shown that 
    \begin{equation}\label{mil4:theo:eq:theta0_psi_first}
        \left[\Theta_0 + \Psi\right](x, k) \approx \left[\Theta_0 + \Psi\right](x\ped{init}, k)\cos(ks(x))
    \end{equation}
    if we keep neglecting the driving force. For fixed $x$, the $k$-modes that correspond to extrema follow a harmonic relationship $k_n= \pi/s(x)$. The crests and troughs in $\Theta_\ell(0,k)$ translate to odd and even peaks in the CMB power spectrum. A rough mapping from Fourier space to harmonic space, $k\mapsto \ell \sim k\chi_*$ (assuming no curvature), implies a coherent series of acoustic peaks in $C(\ell)\leftsquigarrow  \cos^2{(k\sh)}$ at $\ell_n\approx n\ell\ped{pk}$, where $\ell\ped{pk}\equiv \pi \chi_*/\sh$. In the limit where $k\sh \ll 1$, the perturbation is frozen into its initial state. That is, for the largest scales in the CMB anisotropy where $\ell\ll \chi_*/\sh$, there should be no remnants of primordial acoustic oscillations. 

    The positions of these extrema reflects the state of the mode at recombination. With ``state'' is (loosly) meant whether a $k$-mode has succumb to gravitational pressure (maximum compression) or if the radiation pressure managed to overtake (maximum decompression), or somewhere between these extrema. We stress that the onset of such acoustic oscillations occurs after a mode has entered the horizon. The first crest in the CMB spectrum represents the mode that reached maximum compression of the primordial photon-baryon plasma just as said plasma decoupled. The subsequent peak is then the mode that compressed once inside potential wells, then rarefied before recombination. The third peak represents the mode that underwent a complete cycle of compression and decompression before being caught at its maximal compression at the surface of last scattering. The troughs of the CMB power spectrum appear at modes that are neither compressed nor decompressed at this time. 

    The peak locations are sensitive to the presence of curvature. We rewind a little in order to study the effect of curvature. The angular size $\theta$ of a spatial extent $\lambda\ped{phys}$ is $\theta\approx\lambda\ped{phys}/d_A$, where $d_A$ is the angular diameter distance from~\cref{mil1:theo:sec:distances}. The substitution $\lambda\ped{phys}  \dashrightarrow \sfrac{a}{k}$ and~\cref{mil1:theo:eq:dA_of_x} gives $\theta \leftrightsquigarrow  k^{-1}/r(\chi) $, where $r(\chi)$ is defined in~\cref{mil1:theo:eq:r_of_chi}. Now, the angular size of the sound horizon is $\varsigma(x) = s(x)/(r\circ\chi )(x)$, and evaluated at decoupling gives $\varsigma_*\equiv \varsigma(x_*)=\sh/r(\chi_*)$. This angular scale describes the typical extent of a large clump in a CMB temperature map. Now, the lowest angular wavenumber to peak in the anisotropy spectrum should be close to $\ell\ped{pk}\equiv \sfrac{\pi}{\varsigma_*}$, hence $\ell\ped{pk}\propto r(\chi_*) $. Studying~\cref{mil1:theo:eq:r_of_chi} we can see that the this lenght is smaller in a closed universe ($\OmgK < 0$) than in a flat one ($\OmgK = 0$), and the opposite for an open universe ($\OmgK > 0$), i.e.~$\ell\ped{pk}\ap{closed}< \ell\ped{pk}\ap{flat}<\ell\ped{pk}\ap{open}$.\footnote{One arrives at the same conclusion from a geometrical analysis of the perturbation scales (angular size) in the three different realisations of the background geometry.} Thus, the position of the first peak (and second and third, for that matter) can be used to constrain the curvature density $\OmgK$. Since the Eucledian analysis predicts $\ell\ped{pk}\ap{flat} = \pi \chi_*/s_*$, if the actual (observed) peak is shifted e.g.~towards higher multipoles (usually to the left), one could argue that the universe must be hyperbolic. However, one need keep in mind the potential parameter degeneracy and the ambiguity of our mapping from Fourier space to spherical harmonic space. %It is advantageous to use the first few peaks 
    leads to
    
    \textcolor{red}{COOL SYMBOLS:} $\sphericalangle $: spherical angle,$\quad \leadsto $: leads to $\quad\pitchfork$: proper intersection $\quad \vdash$: reduces to


    \textcolor{red}{LITTE O-NOTATION:} $f\sim g \Leftrightarrow (f-g)\in o(g)$
    
    
    % The Eucledian analysis predicts $\ell\ped{pk}\ap{flat} = $
    
    % The Eucledian universe implies simply $\varsigma_*\ap{flat} = \sh/\chi_*$. In the case of an open universe, $\varsigma_*\ap{open} =\varsigma_*\ap{flat} \alpha/\sinh{\alpha} $, where $\alpha =\sqrt{\abs{\kappa}}\chi_* $, and so $\varsigma_*\ap{open}$
    % In the case of an open universe ($\OmgK > 0$ in~\cref{mil1:theo:eq:r_of_chi}), 



    % A rough mapping $k\to \ell/\eta_0$ lets us predict the position of the peaks in $C(\ell) \supset \abs{\cos{ks(x)}}^2$,
    % \begin{equation}
    %     \ell_n \sim n \cross \pi\frac{\eta_0}{\sh}\, ; \quad n\in \mathbb{N}.
    % \end{equation}
    % Likewise, the troughs will be positioned at
    % \begin{equation}
    %     \ell\ped{tr} \sim \left(n+\frac{1}{2}\right)\cross \pi\frac{\eta_0}{\sh}\, ; \quad n\in \mathbb{N}.
    % \end{equation}

    
    \paragraph{Baryon loading}
    We consider the effect of baryon loading. The slightly less simplified analysis gives
    \begin{equation}
        \left[\Theta_0 +(1+R) \Psi\right](x, k) \approx \left[\Theta_0 + (1+R)\Psi\right](x\ped{init}, k)\cos(ks(x)) 
    \end{equation}
    % \begin{equation}
    % \begin{split}
    %     \left[\Theta_0 + \Psi\right](x_*, k) \approx& \left[\Theta_0 + (1+R)\Psi\right](x\ped{init}, k)\cos(ks(x)) \\
    %     &- R(x_*)\Psi(x_*, k),
    % \end{split}
    % \end{equation}
    where $R(x) = \sfrac{3}{4}\Omega\ped{b0} \eu[x]/\Omega\ped{\gamma 0}$ as before. The offsat oscillations in the effective tempeature $\left[\Theta_0 +\Psi\right](x, k)$ are visible in $C(\ell)$ as variations in the amplitudes of the peaks. The presence of baryons has the effect that odd peaks are enhanced, whilst even peaks are suppressed. 

    The physical understanding is that more baryons loads down the plasma in a potential well, giving rise to asymmetric oscillations. For the effect to take place, the amount of baryons in the universe has to be such that their gravitational pull is non-negligable compared to that of the dark matter. In other words, the effect of baryon loading is strongly scaled with the relative composition of normal matter in the universe (e.g.~$\Omega\ped{b0}/\OmgM$).
    
    \paragraph{Radiation driving} 
    What actually is producing the acoustic waves
    


    \paragraph{Diffusion damping}
    Recombination did not occur instantaneously, and as such, the photons bounce around the baryons during the short period it takes for the universe to recombine. Consider a photon in a sea of electrons. Should a random walk take said photon from one region of temperature perturbation scale to another, the photon will try to blend in with other photons, resulting in a mixed temperatures at the scales in question. For fluctuations with wavelength smaller than the mean comoving distance a photon travels of order
    \begin{equation}
        \lambda_D \sim \lambda\ped{mfp}\sqrt{n_e\sigma\ped{T}H^{-1}} = \frac{1}{\sqrt{n_e(x)\sigma\ped{T}\Hp(x) \eu[x]}},
    \end{equation}
    hot and cold photons will mix. As a result, the temperature perturbations are washed out for $k\gtrsim  k_D = 2\pi/\lambda_D$. We expect to see this as a \textit{damping tail} in the CMB power spectrum at particularly small scales. %The mean free path of the photons $\lambda\ped{mfp}$ grows extremely fast in this era, so at some point immediately beyond the last scattering surface, 


    

    



\subsubsection{Matter power spectrum}
    The dimensionless function
    \begin{equation}\label{mil4:theo:eq:Delta_s_of_x_k}
        \Delta_s (x,k) = \delta_s(x,k) - \frac{3(1+w_s)\Hp(x)}{ck}u_s(x,k)
    \end{equation}
    is a gauge-invariant density perturbation used to measure the linear power spectrum of species $s$ at time $x$ and comoving wavenumber $k$. For the total (linear) matter power spectrum $P\ped{m}(x,k)$, we use
    \begin{equation}
    \begin{split}
        \delta\ped{m}(x,k) &= \frac{\delta\ped{c}(x,k) \Omega\ped{c}(x) + \delta\ped{b}(x,k) \Omega\ped{b}(x)}{\OmgM[](x)} \\
        u\ped{m}(x,k) &= \frac{u\ped{c}(x,k) \Omega\ped{c}(x) + u\ped{b}(x,k) \Omega\ped{b}(x)}{\OmgM[](x)}
    \end{split}
    \end{equation}
    in~\cref{mil4:theo:eq:Delta_s_of_x_k} to find
    \begin{equation}\label{mil4:theo:eq:matter_power_spectrum}
        P\ped{m}(x,k) = \abs{\Delta\ped{m}(x,k)}^2 P_\mathcal{R}(k).
    \end{equation}
    We focus on this spectrum today, $P\ped{m0}(k) \equiv P\ped{m}(x\!=\!0,k)$. In short, this function is broad and has a turnover at $k\simeq k\ped{eq}$, i.e.~$\max{P\ped{m0}(k)} \simeq P\ped{m0}(k\ped{eq})$. As density perturbations hardly grow inside the horizon in the radiation-dominated era, the small-scale power spectrum is suppressed relative to its large-scale counterpart. The largest scale for which we get suppression is represented by the $k$-mode that enters the horizon at matter-radiation equality---hence the peak at the equality scale.

    \citet{DodelsonBook} shows that at late times, the approximation
    \begin{equation}\label{mil4:theo:eq:dimless_matter_power_spectrum}
        \Delta\ped{m}(x,k) = \frac{c^2k^2 \eu[x]}{\sfrac{3}{2} \OmgM H_0^2 } \Phi(x,k)\,; \quad x \gtrsim x\ped{\Lambda} ,\quad  k \gg \Hp(x),
    \end{equation}
    suffices when neglecting massive neutrinos. Thus, for the power spectrum today, we expect $P\ped{m0}(k\!\ll\!k\ped{eq}) \propto k^4P_\mathcal{R}(k) \propto k^{\ns}$. 

    \dots

    A key feature of inflation is that it stretches the quxantum fluctuations to cosmological scales, resulting in a nearly scale-invariant power spectrum (see~\cref{mil4:theo:eq:Delta_R_of_k}, where $\ns\approx 1$). 

    We consider a single perturbation (a peak) of normal and matter and relativistic matter in real space at early times. Once photons decoupled from baryons and travelled freely, a spherical shell of baryonic matter surrounding our peak is dragged somewhat out from the sound horizon at decoupling before it remains in place. As time goes by, CDM starts to cluster around this shell. The linearity of the perturbations lets us imagine many such peaks with equally large shells of matter surrounding them. In Fourier space, this corresponds to oscillatory behaviour in $k$ of scale $\sim \sh$. Thus, we expect to see small oscillations in the small-scale part of the total matter power spectrum.



    
    
    % That is to say, in the radiation-dominated era, radiation pressure hindered growth of overdensities, resulting in a relatively flat matter power spectrum.




    