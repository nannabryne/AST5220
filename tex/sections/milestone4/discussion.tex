% !TEX root = ../../main.tex

% ---------------------------------------
% labels: \label{mil4:disc:[type]:[name]}
% ---------------------------------------
% PRESENT TENSE


% \subsubsection{Transfer functions}\label[sec]{mil4:disc:sec:Theta_ell_of_k}

The transfer functions in~\cref{mil4:res:fig:transfer} describe how the primordial density fluctuations at a particular angular wavenumber $\ell$ depend on comiving wavenumber $k$. The LOS integration encapsulates complex physics introduced by the source function. For starters, it provides an image of the relationship between $k$ and $\ell$ which is not one-to-one. For example, $\ell=200$ covers physics at wavenumbers $0.02\unit{Mpc^{-1}} \lesssim k/h \lesssim 0.05\unit{Mpc^{-1}}$.

The multipole numbers considered in~\cref{mil4:res:fig:transfer} deserve some attention. Note that~\cref{mil4:res:fig:CMB_power} is complementary to this discussion. $\ell=6$ probes large-scale physics where $C(\ell) \propto \ell(\ell+1)$, and by extension, the initial conditions. It is asymmetric around the $k$-axis, as opposed to the higher multipoles we see in the upper panel of~\cref{mil4:res:fig:transfer}. This ``lift'' is due to the late ISW effect (c.f.~\cref{mil3:res:fig:gravitational_potential} at $x>x_\Lambda$); without DE domination the prediction is $\Theta_{\ell\lesssim 30}(0, k)\propto j_\ell(k\eta_0)$, which is asymptotically symmetric. $\ell=100$ translates to $k$-modes that entered the horizon in or around the period between matter-radiation equality and recombination. These scales did not manage a complete compression in the primordial plasma before decoupling, but rather were caught in states of increasing photon overdensities, i.e.~compressing in potential wells resulting in higher photon temperature.  

We see that the higher multipoles correspond roughly to angular peaks in the CMB power spectrum, in particular the first $\ell_1 \approx 200$, the second $\ell_2 \approx 500$ and the fourth $\ell_4\approx 1000$. The first represents a compression, evident from the direction of the onset of oscillations in the transfer function, or simply from its encapsulation of the $k_1$-mode. The second and fourth peak are decompressions, which we can see from similar arguments.\footnote{Recall that $k_n=nk_1=n\pi/\sh$.} 



% The photon temperature transfer functions $\Theta_\ell(k)$ describe how the primordial density fluctuations at a particular comoving wavenumber $k$ evolve over time and contribute to the temperature fluctuations observed in the CMB at a particular angular scale $\ell$. They encapsulate the complex interplay between the expansion of the Universe, the growth of density perturbations, the physics of photon-matter interactions, and the subsequent propagation of CMB photons to us.


% \subsubsection{Power spectra}\label[sec]{mil4:disc:sec:C_of_ell_and_P_of_k}

\subsubsection{CMB anisotropy}\label[sec]{mil4:disc:sec:C_of_ell}
We turn to the main result of this project in~\cref{mil4:res:fig:CMB_power}. The Sachs-Wolfe and Doppler contributions to the CMB anisotropy are by far most significant in order to reproduce the observed spectrum. We notice the Sachs-Wolfe plateau for small $\ell$. The polarisation term is negligible in our model. 

The power at the SW plateau is sensitive to the primordial amplitude and substantiate the fiducial value $\As \sim 10^{-9}$. The tilt in the purely SW contribution ($\mathcal{D}(\ell)^\mathrm{SW} > \mathcal{D}(\ell')^\mathrm{SW} \,\forall \,\ell < \ell' \lesssim 30$) is there because the scalar spectral index is less than 1. That is, $\ns\approx 0.96$ in the primordial power spectrum is also a result of a fitting analysis at these scales---corresponding to angular separations on the sky of $\sim 90^\circ$---in the CMB anisotropy.



The increasing significance of the integrated Sachs-Wolfe effect for smaller $\ell$ is visible in the plot. This is the late-time ISW large-scale effect kicking in as dark energy starts to dominate and thereby affect the gravitational potential in such a way that \textcolor{blue}{help}


We discuss below the different contributions to the CMB anisotropy in a systematical manner.

\dots

\dots 

\dots 

\dots

%  PEAKS
% (ANGULAR PEAKS:) 
\paragraph{Angular peaks}
As we know, modes caught in states of extrema at the time of recombination make up the angular peaks---the peaks in the CMB power spectrum. We see that the mode that compressed once inside potential wells prior to recombination corresponds to angular wavenumber $\ell\sim 200$, that is an angular size of $\sim 1^\circ$. The second peak is significantly suppressed relative to the first and third peak. As the odd peaks represent compressions in the photon-baryon plasma, this result gives a clear indication of a preeminent contribution from CDM to the total matter.  



% MATTER 

% \paragraph{Matter inhomogeneity}
\subsubsection{Matter inhomogeneity}
Oscillations in the matter power spectrum in~\cref{mil4:res:fig:matter_power} are only present for the small-scale modes ($k>k\ped{eq}$), as we expected. The peak at $k\simeq k\ped{eq}$ represents the characteristic scale of structures that entered the horizon during the radiation-dominated era. It signifies the transition from the regime where radiation pressure dominates (with suppressed growth; $k>k\ped{eq}$) to the regime where gravitational instability becomes dominant (with enhanced growth; $k<k\ped{eq}$). \textcolor{blue}{CHECK THIS!} 

The acoustic wiggles reflect the shell-like clustering of matter at sound horizon scales. 

The graph's poor coherence with observational data for intermediate- and small-scale modes may be due to the neglection of neutrinos and other ingredients. Including neutrinos in only the background (setting $N\ped{eff}> 0$) would give very different matter overdensities at late times. This could affect the resulting $P\ped{m0}(k)$. However, at very small scales, we know that the linear theory is not sufficient to describe the total matter power spectrum.
