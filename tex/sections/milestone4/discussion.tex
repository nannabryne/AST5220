% !TEX root = ../../main.tex

% ---------------------------------------
% labels: \label{mil4:disc:[type]:[name]}
% ---------------------------------------
% PRESENT TENSE


The transfer functions in~\cref{mil4:res:fig:transfer} 



We turn to the main result of this project in~\cref{mil4:res:fig:CMB_power}. The Sachs-Wolfe and Doppler contributions to the CMB anisotropy are by far most significant in order to reproduce the observed spectrum. We spot the Sachs-Wolfe plateau for small $\ell$. The polarisation term is negligible in our model. 

The increasing significance of the integrated Sachs-Wolfe effect for smaller $\ell$ is visible in the plot. This is the late-time ISW large-scale effect kicking in as dark energy starts to dominate and thereby affect the gravitational potential in such a way that \textcolor{blue}{help}


We discuss below the different contributions to the CMB anisotropy in a systematical manner.

\dots

\dots 

\dots 

\dots

%  PEAKS
(ANGULAR PEAKS:) As we know, modes caught in states of extrema at the time of recombination make up the angular peaks---the peaks in the CMB power spectrum. We see that the mode that compressed once inside potential wells prior to recombination corresponds to an angular wavenumber $\ell\sim 200$. 