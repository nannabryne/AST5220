% !TEX root = ../../main.tex

% ---------------------------------------
% labels: \label{mil4:disc:[type]:[name]}
% ---------------------------------------
% PRESENT TENSE
\dots

\subsubsection{Transfer functions}\label[sec]{mil4:disc:sec:Theta_ell_of_k}
The transfer functions in~\cref{mil4:res:fig:transfer} describe how the primordial density fluctuations at a particular angular wavenumber $\ell$ depend on comiving wavenumber $k$. The LOS integration encapsulates complex physics introduced by the source function.

% The photon temperature transfer functions $\Theta_\ell(k)$ describe how the primordial density fluctuations at a particular comoving wavenumber $k$ evolve over time and contribute to the temperature fluctuations observed in the CMB at a particular angular scale $\ell$. They encapsulate the complex interplay between the expansion of the Universe, the growth of density perturbations, the physics of photon-matter interactions, and the subsequent propagation of CMB photons to us.


\subsubsection{Power spectra}\label[sec]{mil4:disc:sec:C_of_ell_and_P_of_k}

We turn to the main result of this project in~\cref{mil4:res:fig:CMB_power}. The Sachs-Wolfe and Doppler contributions to the CMB anisotropy are by far most significant in order to reproduce the observed spectrum. We spot the Sachs-Wolfe plateau for small $\ell$. The polarisation term is negligible in our model. 

The increasing significance of the integrated Sachs-Wolfe effect for smaller $\ell$ is visible in the plot. This is the late-time ISW large-scale effect kicking in as dark energy starts to dominate and thereby affect the gravitational potential in such a way that \textcolor{blue}{help}


We discuss below the different contributions to the CMB anisotropy in a systematical manner.

\dots

\dots 

\dots 

\dots

%  PEAKS
(ANGULAR PEAKS:) As we know, modes caught in states of extrema at the time of recombination make up the angular peaks---the peaks in the CMB power spectrum. We see that the mode that compressed once inside potential wells prior to recombination corresponds to angular wavenumber $\ell\sim 200$, that is an angular size of $\sim 1^\circ$. The second peak is significantly suppressed relative to the first and third peak. As the odd peaks represent compressions in the photon-baryon plasma, this result gives a clear indication of a preeminent contribution from CDM to the total matter.  



% MATTER 
(MATTER POWER SPECTRUM)

Oscillations in the matter power spectrum in~\cref{mil4:res:fig:matter_power} are only present for the small-scale modes ($k>k\ped{eq}$), as we expected. The peak at $k\simeq k\ped{eq}$ represents the characteristic scale of structures that entered the horizon during the radiation-dominated era. It signifies the transition from the regime where radiation pressure dominates (with suppressed growth; $k>k\ped{eq}$) to the regime where gravitational instability becomes dominant (with enhanced growth; $k<k\ped{eq}$). \textcolor{blue}{CHECK THIS!} 

The acoustic wiggles reflect the shell-like clustering of matter at sound horizon scales. 

The graph's poor coherence with observational data for intermediate- and small-scale modes may be due to the neglection of neutrinos and other ingredients. Including neutrinos in only the background (setting $N\ped{eff}> 0$) would give very different matter overdensities at late times. This could affect the resulting $P\ped{m0}(k)$. However, at very small scales, we know that the linear theory is not sufficient to describe the total matter power spectrum.
