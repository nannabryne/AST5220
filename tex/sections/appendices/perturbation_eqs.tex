% !TEX root = ../../main.tex

% --------------------------------------
% labels: \label{app:pert:[type]:[name]}
% --------------------------------------


This appendix contains equations relevant for~\cref{sec:mil3}.

We let $' \equiv \dv{}{x}$ in the equations below. Additionally, we introduce $\ckH = \ckH(x, k)\equiv \frac{ck}{\Hp(x)}$ as a shorthand notation.


\subsection{The full system}\label[app]{app:pert:sec:full_system}

% The full system $\vec{Y}=(\Phi, \delta\ped{c}, \delta\ped{b}, u\ped{c}, u\ped{b}, \Theta_0, \Theta_1, \dots, \Theta_{\ell\ped{max}})$ \dots

The metric perturbations evolve as follows:
\begin{subequations}\label{app:pert:eq:metric_perturbations}
\begin{align} 
    \Phi' &= \Psi - \frac{\ckH^2}{3} \Phi + \frac{H_0^2}{2\Hp^2\eu[2x]}\left\{ \left(\Omega\ped{c 0}\delta\ped{c} + \Omega\ped{b 0}\delta\ped{b} \right)\eu[x]+4\Omega\gped{\textgamma 0}\Theta_0\right\} \label{app:pert:eq:dPhidx}\\
    \Psi &= - \Phi - \frac{12H_0^2}{c^2k^2\eu^{2x}} \Omega\gped{\textgamma 0}\Theta_2 \label{app:pert:eq:Psi}
\end{align}
\end{subequations}
Note that~\cref{app:pert:eq:Psi} is not a differential equation. The perturbations to normal matter ($s=\mathrm{c,b}$) are the following:
\begin{subequations}\label{app:pert:eq:matter_perturbations}
\begin{align}
    \delta_s'&=  \ckH u_s - 3\Phi'\label{app:pert:eq:ddelta_sdx}  \\
    u_s' &= - u_s - \ckH\Psi + \kroneckerdelta{s\mathrm{b}}  \frac{\tau'}{R} \left(3\Theta_1 +u\ped{b}\right)\label{app:pert:eq:du_sdx} 
\end{align}
\end{subequations}
\begin{subequations}\label{app:pert:eq:photon_multipoles}
The monopole and dipole evolves as:
\begin{align}
    \Theta_0' &= -\ckH \Theta_1 - \Phi ' \label{app:pert:eq:dTheta0dx}\\
    \Theta_1' &= \frac{\ckH}{3} \left[ \Theta_0 - 2\Theta_2 + \Psi \right] + \tau'\left[\Theta_1 + \frac{u\ped{b}}{3} \right] \label{app:pert:eq:dTheta1dx}
\end{align}
For the quadrupole and higher multipoles, we have:
\begin{align}
\Theta_\ell' = \begin{cases}
    % \frac{\ckH}{5} \left[ 2 \Theta_1 -  3 \Theta_3 \right] +\frac{9}{10} \tau' \Theta_2  & \ell = 2 \\ 
    \frac{\ckH}{2\ell + 1} \left[ \ell \Theta_{\ell-1}-  (\ell+1) \Theta_{\ell+1} \right] + \frac{9}{10}\tau ' \Theta_\ell & \ell = 2 \\ 
    \frac{\ckH}{2\ell + 1} \left[ \ell \Theta_{\ell-1}-  (\ell+1) \Theta_{\ell+1} \right] + \tau ' \Theta_\ell  & 2< \ell < \ell\ped{max} \\
    \ckH \Theta_{\ell-1} - \frac{c(\ell+1)}{\Hp \eta}\Theta_{\ell} + \tau' \Theta_{\ell}& \ell=\ell\ped{max}
\end{cases} \label{app:pert:eq:dThetaelldx}
\end{align}
\end{subequations}




\subsection{Tight coupling regime}\label[app]{app:pert:sec:tight_coupling}
The tight coupling regime considers~\cref{app:pert:eq:metric_perturbations},~\cref{app:pert:eq:dTheta0dx} and~\cref{app:pert:eq:matter_perturbations} except for~\cref{app:pert:eq:du_sdx} for $s=\mathrm{b}$. The following equations substitute the missing derivatives $\Theta_1'$ and $u\ped{b}'$:
\begin{subequations}\label{app:pert:eq:tight_coupling}
\begin{align}
    u\ped{b}' &= \frac{q-Ru\ped{b} + \ckH(-\Theta_0+2\Theta_2)}{1+R} - \ckH\Psi \\
    \Theta_1' &= \frac{1}{3}\left(q-u\ped{b}'\right)
\end{align}
$q$ is given by:
\begin{equation}
\begin{split}
    q = &\left[\tau' (1+R) - R\left(1-\frac{\Hp'}{\Hp}\right)\right]^{-1} \cross \\
    &\Bigg\{ -\left[\tau''(1+R) - \tau'(1-R) \right] \left(3\Theta_1 + u\ped{b}\right) \\
    &\,  - \ckH R\left[  \Psi - \left(1-\frac{\Hp'}{\Hp}\right)\left(-\Theta_0+ 2\Theta_2\right)+ \Theta_1'\right]
    \Bigg\}
\end{split}
\end{equation}
\end{subequations}
The higher multipoles are given by~\cref{app:pert:eq:initial_conditions_Theta2ell}
% We dismiss the photons multipoles higher than the quadrupole. We need the quadrupole to solve the coupled set of equations, but we will use~\cref{app:pert:eq:init_Theta2} to compute this quantity instead of a differential equation.


\subsection{Initial conditions}\label[app]{app:pert:sec:initial_conditions}

Let $\Psi\ped{init}\equiv\Psi(x\ped{init}, k)$ and $k\eta(x\ped{init}) \ll 1$. Further, we use that $\Psi\ped{init}=-\sfrac{2}{3}$ and the following scalar quantities at $x=x\ped{init}$:
\begin{subequations}\label{app:pert:eq:initial_conditions}
\begin{align}
    % \Psi(x\ped{init}, k)     &= \Psi\ped{init} =-\frac{2}{3}     \label{app:pert:eq:init_Psi}\\
    \Phi(x\ped{init}, k)     &= -\Psi\ped{init}                                 \label{app:pert:eq:init_Phi}\\
    \delta_s(x\ped{init}, k) &= -\frac{3}{2} \Psi\ped{init}                   \,; &&s=\mathrm{c,b} \label{app:pert:eq:init_delta}\\
    u_s(x\ped{init}, k)      &= -\frac{\ckH(x\ped{init}, k)}{2} \Psi\ped{init}\,; &&s=\mathrm{c,b} \label{app:pert:eq:init_u}
\end{align}
\end{subequations}
The first two photon multipoles are given by the following:
\begin{subequations}\label{app:pert:eq:initial_conditions_Theta01}
\begin{align}
    \Theta_0(x\ped{init}, k) &= -\frac{1}{2}\Psi\ped{init}        \label{app:pert:eq:init_Theta0}\\
    \Theta_1(x\ped{init}, k) &= +\frac{\ckH(x\ped{init}, k)}{6}\Psi\ped{init}     \label{app:pert:eq:init_Theta1}
\end{align}
\end{subequations}
For the quadrupole and the remaining multipoles, the following expressions hold at early times:
\begin{subequations}\label{app:pert:eq:initial_conditions_Theta2ell}
\begin{align}
    \Theta_2(x, k) &= -\frac{4\ckH(x, k)}{9\tau'(x)}\Theta_{1}(x, k)\label{app:pert:eq:init_Theta2} \\
    \Theta_\ell(x, k) &= -\frac{\ell}{2\ell+1} \frac{\ckH(x, k)}{\tau'(x)} \Theta_{\ell-1}(x, k)\,; \quad \ell > 2 \label{app:pert:eq:init_Thetaell}
\end{align}
\end{subequations}










% \please

% \noindent\colorbox{orange}{FULL SYSTEM:}
% \par \textcolor{blue}{\underline{Metric perturbations}}
% \begin{subequations}\label{app:pert:eq:metric_perturbations2}
% \begin{align}
%     \dv{\Phi}{x} &= \Psi - \frac{c^2 k^2}{3\Hp^2} \Phi + \frac{H_0^2}{2\Hp^2\eu[2x]}\left\{ \left(\Omega\ped{c 0}\delta\ped{c} + \Omega\ped{b 0}\delta\ped{b} \right)\eu[x]+4\Omega\gped{\textgamma 0}\Theta_0\right\} \\
%     \Psi &= - \Phi - \frac{12H_0^2}{c^2k^2\eu^{2x}} \Omega\gped{\textgamma 0}\Theta_2 
% \end{align}
% \end{subequations}

% \par \textcolor{blue}{\underline{Matter perturbations}}
% \begin{subequations}\label{app:pert:eq:matter_perturbations2}
% \begin{align}
%     \dv{\delta\ped{c, b}}{x} &=  \ckH u\ped{c, b} - 3\dv{\Phi}{x}  \\
%     \dv{u\ped{c, b}}{x} &= - u\ped{c, b} - \ckH\Psi + \mathbf{1}\ped{b} \dv{\tau}{x} \frac{3\Theta_1 +u\ped{b}}{R(x)} 
% \end{align}
% \end{subequations}

% \par \textcolor{blue}{\underline{Photon temperature perturbations}}
% \begin{equation}\label{app:pert:eq:photontemp_perturbations}
% \dv{\Theta_\ell}{x} = \begin{cases}
%     - \ckH\Theta_{\ell+1}- \dv{\Phi}{x} & \ell =0\\
%     \frac{ck}{3\Hp} \left( \Theta_{\ell-1} - 2\Theta_{\ell+1} + \Psi\right) + \dv{\tau}{x}\left[\Theta_{\ell} + \frac{1}{3}u\ped{b} \right] & \ell =1  \\
%     \frac{ck}{(2\ell +1)\Hp} \left(\ell \Theta_{\ell-1}- (\ell + 1)\Theta_{\ell+1} \right) +\frac{9}{10} \dv{\tau}{x}\Theta_{\ell} &  \ell =2 \\
%     \frac{ck}{(2\ell +1)\Hp} \left(\ell \Theta_{\ell-1}- (\ell + 1)\Theta_{\ell+1} \right) + \dv{\tau}{x}\Theta_\ell  & 3 \leq \ell < \ell\ped{max} \\
%     \ckH \Theta_{\ell-1}- \frac{c(\ell+1)}{\Hp \eta}\Theta_{\ell} + \dv{\tau}{x}\Theta_\ell &  \ell = \ell\ped{max}
% \end{cases}
% \end{equation}


% \begin{equation}
% \begin{split}
%     \dv{\Theta_\ell}{x} = &\frac{ck\left(\ell \Theta_{\ell-1} - (\ell+1)\Theta_{\ell+1}\right)}{(2\ell +1) \Hp}  + \dv{\tau}{x} \Theta_\ell \left(1-\frac{\delta_{\ell,2}}{10}\right) + \mathcal{K}
% \end{split}
% % \left[\Theta_\ell + \frac{\delta_{\ell,1} u\ped{b}}{3} - \frac{\delta_{\ell,2}\Theta_\ell}{10} \right] + \mathcal{K}
% \end{equation}
% \begin{equation}
%     \mathcal{K} = \begin{cases}
%         -\dv{\Phi}{x} &\ell = 0 \\
%         \frac{ck}{3\Hp}\Psi + \dv{\tau}{x}\frac{u\ped{b}}{3} & \ell =1 \\
%         0 & \ell \leq 2
%     \end{cases}
% \end{equation}

% \begin{equation}
% \begin{split}
%     \dv{\Theta_\ell}{x} = &\frac{ck\left[\ell \Theta_{\ell-1} - (\ell+1)\Theta_{\ell+1}\right]}{(2\ell +1) \Hp}  + \dv{\tau}{x} \Theta_\ell \\%\left(1-\frac{\delta_{\ell,2}}{10}\right) \\
%     &- \delta_{\ell,0}\dv{\Phi}{x} + \delta_{\ell,1}\left[\frac{ck}{3\Hp}\Psi + \dv{\tau}{x}\frac{u\ped{b}}{3}\right] - \delta_{\ell,2} \dv{\tau}{x}\frac{\Theta_\ell}{10}
% % \end{split}
% % \end{equation}
% \begin{subequations}
% \par \textcolor{blue}{$\ell = 0$ --- FIXME!!} 
% \begin{equation}
%     \dv{\Theta_{\ell= 0}}{x} = \frac{ck}{\Hp}\Theta_{\ell+1}  -\dv{\Phi}{x}
% \end{equation}
% \par \textcolor{blue}{$\ell < \ell\ped{max}$ --- FIXME!!} 
% \begin{equation}
% \begin{split}
%     \dv{\Theta_{0<\ell< \ell\ped{max}}}{x} = &\frac{ck\left[\ell \Theta_{\ell-1} - (\ell+1)\Theta_{\ell+1}+\delta_{\ell,1} \Psi\right]}{(2\ell +1) \Hp}  \\
%     &+ \dv{\tau}{x}\left[\left(1 -\frac{\delta_{\ell,2}}{10}\right)\Theta_\ell+ \frac{\delta_{\ell,1} u\ped{b}}{3}\right] 
% \end{split}
% \end{equation}
% \par \textcolor{blue}{$\ell = \ell\ped{max}$}
% \begin{equation}
%     \dv{\Theta_{\ell = \ell\ped{max}}}{x} = \ckH \Theta_{\ell-1} - \frac{c(\ell+1)}{\Hp \eta}\Theta_{\ell} + \dv{\tau}{x} \Theta_{\ell}
% \end{equation}
% \end{subequations}
% \begin{equation}\label{app:pert:eq:photontemp_perturbations}
% \dv{\Theta_\ell}{x} = \frac{ck}{(2\ell +1)\Hp} \left(\ell \Theta_{\ell-1}- (\ell + 1)\Theta_{\ell+1} \right) + \dv{\tau}{x}\Theta_\ell


% \end{equation}



% \noindent\colorbox{orange}{TIGHT COUPLING REGIME:}

% \begin{subequations}\label{app:pert:eq:tight_regime}

% \begin{equation}
% \begin{split}
%     q = &\left[\dv{\tau}{x}~(1+R) - R\left(1-\frac{1}{\Hp}\dv{\Hp}{x}\right)\right]^{-1} \cross \\
%     \,&\Bigg\{ \left[ \dv[2]{\tau}{x}~(1+R) - \dv{\tau}{x}~(1-R) \right] (3\Theta_1 + u\ped{b}) \\
%     \,&  - \ckH R\left[  \Psi + \left(1-\frac{1}{\Hp}\dv{\Hp}{x}\right)\left(-\Theta_0+ 2\Theta_2\right)- \dv{\Theta_0}{x}\right]
%     \Bigg\}
% \end{split}
% \end{equation}
% \begin{equation}
% \begin{split}
%     \dv{u\ped{b}}{x} =& \frac{q-Ru\ped{b} + \ckH(-\Theta_0+2\Theta_2)}{1+R} - \ckH\Psi 
% \end{split}
% \end{equation}
% \begin{equation}
% \begin{split}
%     \dv{\Theta_1}{x} =& \frac{1}{3}\left(q-\dv{u\ped{b}}{x}\right)
% \end{split}
% \end{equation}
% \end{subequations}


% \begin{subequations}\label{app:pert:eq:initial_conditions}
% \begin{align}
%     \Psi     &= -\frac{2}{3}     \label{app:pert:eq:init_Psi}\\
%     \Phi     &= -\Psi         \label{app:pert:eq:init_Phi}\\
%     \delta\ped{c, b}&= -\frac{3}{2} \Psi     \label{app:pert:eq:init_delta}\\
%     u\ped{c, b}     &= -\frac{ck}{2\Hp} \Psi \label{app:pert:eq:init_u}\\
%     \Theta_0 &= -\frac{1}{2}\Psi \label{app:pert:eq:init_Theta0}\\
%     \Theta_1 &= +\frac{ck}{6\Hp}\Psi\label{app:pert:eq:init_Theta1}\\
%     \Theta_2 &= -\frac{8ck}{15\Hp} \left(\dv{\tau}{x}\right)^{-1}\Theta_1\label{app:pert:eq:init_Theta2}\\
%     \Theta_\ell &= -\frac{\ell}{2\ell+1} \frac{ck}{15\Hp} \left(\dv{\tau}{x}\right)^{-1}\Theta_{\ell-1}\label{app:pert:eq:init_Thetal}
% \end{align}
% \end{subequations}


% \noindent\colorbox{orange}{INITIAL CONDITIONS:}
% \begin{subequations}\label{app:pert:eq:initial_conditions}
% \begin{align}
%     \Psi     &= -\frac{2}{3}     \label{app:pert:eq:init_Psi}\\
%     \Phi     &= -\Psi         \label{app:pert:eq:init_Phi}\\
%     \delta\ped{c, b}&= -\frac{3}{2} \Psi     \label{app:pert:eq:init_delta}\\
%     u\ped{c, b}     &= -\frac{ck}{2\Hp} \Psi \label{app:pert:eq:init_u}\\
%     \Theta_\ell &=\begin{cases}
%         -\frac{1}{2}\Psi     & \ell = 0  \\
%         +\frac{ck}{6\Hp}\Psi  & \ell = 1 \\ 
%         -\frac{8ck}{15\Hp\dv*{\tau}{x}}\Theta_1    & \ell = 2 \\ 
%         -\frac{\ell}{2\ell+1} \frac{ck}{\Hp\dv*{\tau}{x}} \Theta_{\ell-1} &\ell \geq 3
%     \end{cases} \label{app:pert:eq:init_Theta}
% \end{align}
% \end{subequations}

% \sendhelp
