% !TEX root = ../../main.tex

% ---------------------------------------
% labels: \label{mil1:disc:[type]:[name]}
% ---------------------------------------
% PRESENT TENSE

The graphs in~\cref{mil1:res:fig:density_params} show that prior to $x\sim-15$, radiation was the prevailing constituent of matter and energy density in the universe, making $x\ped{init}=-20$ in~\cref{mil1:theo:sec:approximations} a valid choice. The radiation era is followed by an era of matter domination before the universe enters its current epoch where dark energy is the preeminent contributor to the cosmic energy budget. The universe has not yet become overwhelmed by the dark energy, however, and we can see that we are currently in a transitional period between total matter domination and total DE domination. The graphs also show that this transitional period is much quicker than the previous one, and that just before matter-DE equality ($x=x_{\Lambda}$), the universe starts accelerating ($x=x\ped{acc}$). We clearly see the relation between the dark energy suddenly becoming significant and the universe accelerating. 

Our results, as shown in the graphs in~\cref{mil1:res:fig:tests_hubble}, demonstrate that our code gives sensible results when modelling the evolution of the universe. The changes in the density parameters depicted in~\cref{mil1:res:fig:density_params} offer insight into the transitional periods shown in~\cref{mil1:res:fig:hubble_of_x} and~\cref{mil1:res:fig:der_hubble}, which correspond to eras where no single substance dominates the universe. These findings are consistent with existing knowledge of the universe's history. Additionally, we observe that the conformal time cannot be accurately predicted in the same way for later eras, as demonstrated in~\cref{mil1:res:fig:eta_hubble}, exactly as expected (\textcolor{blue}{ref to some section}).

% The graphs in~\cref{mil1:res:fig:tests_hubble} indicate that our code is working. The evolution of the density parameters in~\cref{mil1:res:fig:density_params} explains the deviations around the transitional periods in~\cref{mil1:res:fig:hubble_of_x} and~\cref{mil1:res:fig:der_hubble} as they correspond to eras of not really one dominating substance. As expected, the conformal time cannot be predicted in the same way for the later eras. This we see from~\cref{mil1:res:fig:eta_hubble}.

% According to this model, the radiation-matter equality happened at redshift 3400 and the universe entered its current epoch about 3.5 gigayears ago. This is seen from~\cref{mil1:res:tab:time_of_events}. We also find that the universe is 13.9 gigayears old and that it has been accelerating for almost half of that time, starting at the age of 7.8 gigayears. These numbers differ slightly from what we find in literature, e.g.~the age $t_0=13.78$ in~\citet{DodelsonBook}. The cosmic time seems to be generally a bit too high. It is natural to address the numerical vulnerability that arises when choosing a grid for $x$ and propagates into the computed variables. From~\cref{mil1:res:fig:conformal_cosmic_time} we can see how over 10 gigayears passes between $x=-2$ and $x=0$, even though we started integrating from $x=x\ped{init}=-20$. This allows numerical flaws in $x$ to propagate in an unpredictable fashion into variables like $t(x)$.

In this model, the redshift of 3400 marks the epoch of radiation-matter equality, while the universe entered its current epoch approximately 3.5 gigayears ago as shown in~\cref{mil1:res:tab:time_of_events}. Additionally, our analysis reveals that the universe is estimated to be 13.9 gigayears old and has been accelerating for almost half of that time, starting at the age of 7.8 gigayears. These predictions differ slightly from those reported in the literature, such as an age of $t_0=13.78$ in~\citet{DodelsonBook}. It is sufficient to argue that the main reason for such deviations is the (small) difference in choice of cosmological parameters. However, we would like to address the computational limitations: the choice of grid for $x$ may introduce numerical vulnerabilities that propagate into other variables. For instance, as illustrated in~\cref{mil1:res:fig:conformal_cosmic_time}, over 10 gigayears can pass between $x=-2$ and $x=0$, even though we started integrating from $x=x\ped{init}=-20$.  









%The cosmic time appears to be generally slightly higher in our model. 

%in a somewhat unpredictable fashion for the 
%A plausible explanation could be that this is due to numerics.


\subsubsection{Supernova fitting}
    The comparison of our model's luminosity distance with observational data, as shown in~\cref{mil1:res:fig:lum_dist_vs_z} (ignoring the posterior green graph), suggests that our model could benefit from some adjustments. While the deviations are not too far off, it is clear that there is some room for improvement. However, it is important to note that the discrepancies may not be solely due to the three parameters we chose to study. 
    %Other factors, such as the distribution of dark matter or the effect of cosmic voids, could also play a role in explaining the deviations from observations.

    To further constrain our model, we performed an MCMC analysis and examined the resulting distributions of parameters. The scatter plot in~\cref{mil1:res:fig:omega_space} shows that our model requires an accelerating universe ($\dv*{\Hp}{x}|_{x=x_0} > 0$) with a strictly positive cosmological constant ($\OmgL>0$). We notice that the Planck parameters lie within the 1\textsigma~region for $(\OmgM,\, \OmgL)$, along the flat line. This is not the case for all the parameters, however. 
    
    Interestingly, the data clearly prefer a slightly higher value for the little Hubble parameter $h$ than our fiducial value of 0.67, as shown in the narrow histogram in~\cref{mil1:res:fig:hubble_pdf}. The PDFs of the density parameters $\OmgM$ and $\OmgL$ are much broader, but the algorithm manages to narrow the possibilities down significantly.

    One potential concern is the uncertainty in the curvature parameter $\OmgK$, as shown in~\cref{mil1:res:fig:mcmc_results}. The data seem to favour a negatively curved universe, but the allowed range is quite broad. This may indicate that our model needs further refinement to better account for the effects of curvature. 
    
    Overall, our primitive MCMC analysis provides valuable insights into the constraints on the parameters of our model and highlights areas where further improvements could be made. Other than with observational data from supernovae, there are several ways of constraining the cosmological parameters, such as measuring the CMB anisotropies.~\citet{Planckdata} provides a set of cosmological parameters that are significantly more solid, in the sense that their results are tested and compared thoroughly. This is why we proceed using the fiducials from~\cref{mil1:imp:eq:fiducials} and~\cref{mil1:imp:eq:derived_fiducials}.




    % The scatter plot in~\cref{mil1:res:fig:omega_space} puts constraints on the matter and DE density ($\OmgM$ and $\OmgL)$ which clearly require an accelerating universe ($\D \Hp/\D x|_{x=x_0} > 0$) and a strictly positive cosmological constant ($\OmgL>0$). This is when allowing a curved universe ($\OmgK\in [-1, 1]$). We see that the Planck parameters fit nicely inside the 1\textsigma\, region and on the flat ($\OmgK=0$) line. 

    % The histograms in~\cref{mil1:res:fig:mcmc_results} inform us that the algorithm insists that the little Hubble parameter is in the range between 0.68 and 0.72, that is \textit{not} our fiducial $h=0.67$. The PDFs of the density parameters are much broader, but the algorithm manages to narrow the down the possibilities a great amount on $\OmgM$ and $\OmgL$. There uncertainty in $\OmgK$ is disturbing, but the algorithm seems to favour a negatively curved background.