% --------------------------------------
% labels: \label{mil1:imp:[type]:[name]}
% --------------------------------------
% PAST TENSE


The code we wrote included a class in \verb|C++| representing the background. In particular, this class requires current values of the density parameters $\Omega_{\mathrm{b}0}$, $\Omega_{\mathrm{CDM}0}$ and $\Omega_{\mathrm{K}0}$, the CMB temperature today ($\TCMB$) and the effective neutrino number ($N_\mathrm{eff}$). In addition, the class needs the ``little'' Hubble constant $h=H_0$~$[100\text{~km}\text{~s}^{-1}\text{~Mpc}^{-1}]$. The remaining density parameters are computed as elaborated in \cref{mil1:sec:theo}. Amongst the class methods are functions for computing $\Hp(x)$, $\dv*{\Hp(x)}{x}$, $\dv*[2]{\Hp(x)}{x}$ and $\Omega_{s0}(x)$ for some $x$, as well as code that solves the ODEs for $\eta(x)$ and $t(x)$. Another vital method is the one that yields the luminosity distance $d_L(x)$ for some $x$.

% The specifics of our flat ($\OmgK=0$) model was stolen from \citep{Planckdata} with $h=0.67$, $\TCMB=2.7255$ K, $N_\mathrm{eff}=3.046$, $\Omega_{\mathrm{b}0} = 0.05$, and $\Omega_{\mathrm{CDM}0}= 0.267$. 

The specifics of our model was found from fits to \citep{Planckdata}:
\begin{equation}\label{mil1:imp:eq:fiducials}
    \begin{split}
        \text{Hubble constant:}&& h &=0.67 \\
        \text{CMB temperature:}&& \TCMB &=2.7255\text{~K} \\
        \text{effective Neutrino number:}&& N\ped{eff}&=3.046\\
        \text{baryon density:}&& \Omega\ped{b 0}&=0.05\\
        \text{CDM density:}&& \Omega\ped{CDM 0}&=0.267\\
        \text{curvature density:}&& \OmgK&=0
    \end{split}
\end{equation}
This gave the following derived parameters:
\begin{equation}\label{mil1:imp:eq:derived_fiducials}
    \begin{split}
        \text{photon density:}&& \Omega\gped{\textgamma 0}&=5.51 \cross 10^{-5}\\
        \text{neutrino density:}&& \Omega\gped{\textnu 0}&=3.81 \cross 10^{-5}\\
        \text{DE density:}&& \OmgL&=0.683
    \end{split}
\end{equation}
We evaluated the various quantities over $x \in [-20, 5]$, the same interval for which we numerically solved the ODEs for $\eta(x)$ and $t(x)$ in \cref{mil1:theo:eq:odes_time}, setting $x\ped{init}=-20$ in \cref{mil1:theo:eq:time_initial_conditions}. 

After controlling our model by comparing numerical results to analytical expressions in limit cases, we turned our attention to the observational data from \citep{supernovadata}. The data set is constructed as follows: for each redshift $z_i$, there is an observed luminosity distance $d_L\ap{obs}(z_i)$ and an associated error $\sigma\ped{err}(z_i)$. % for en krise setning skjerp deg Nanna

Subsequently, we wrote a script to perform an MCMC for the parameters $h$, $\OmgM$ and $\OmgK$. Running said script, we compared the computed luminosity distance $d_L(z)$ from a cosmological model (an instance of the class) to the observed luminosity distance $d_L\ap{obs}(z)$ through the \textchi$^2$-function,
\begin{equation}
    \chi^2(h, \OmgM, \OmgK) = \sum_{i=1}^N \frac{ \left( d_L(z_i; h, \OmgM, \OmgK) - d_L\ap{obs}(z_i)\right)^2 }{\sigma\ped{err}^2(z_i)},
\end{equation}
where $N=31$ is the number of data points. The best-fit model was considered as the one for which $\chi^2 = \chi^2\ped{min}$, the lowest number found by the algorithm. A good fit is considered to have $\chi^2/N \sim 1$. 

The MCMC analysis was characterised by a maximum of 10\,000 iterations and the following limitations:
\begin{equation}\label{mil1:imp:eq:mcmc_limits}
    \begin{split}
        0.5 \leq&& h &&\leq 1.5 \\
        0.0 \leq&& \OmgM &&\leq 1.0 \\
        -1.0 \leq&& \OmgK &&\leq 1.0 
    \end{split}
\end{equation}
They were initialised by sample from a uniform distribution in the respective range. Once the code was executed successfully, we discard the first 200 samples, expecting this to be the approximate burn-in period for of the Metropolis MCMC.
